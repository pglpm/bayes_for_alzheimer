%%This article HOWTO last updated 25 February 2002 ---TSNull
\documentclass[titlepage,12pt]{article}

%\usepackage[dvips]{graphicx}
\usepackage[draft,dvips]{graphicx}
%\usepackage[pdftex]{graphicx}
%\usepackage[draft,pdftex]{graphicx}

\usepackage[myheadings]{fullpage}
\usepackage{pmetrika}
\usepackage{pmbib}
\usepackage{submit}

\setcounter{secnumdepth}{3}
%\flushbottom

\usepackage{longtable}

\begin{document}

\begin{titlepage}

\title{HowTo Document: Pmetrika \LaTeX2e\ Style File Package for Psychometrika Authors}


%\markright{\textsc{Version 2 HOWTO}}

\linespacing{1}

\author{Tim Null}

\affil{Technical Editor}

\affil{Psychometrika}



\vspace{36pt}\centerline{\today}\vspace{84pt}\longpage



\comment{Although authors are encouraged to use either the Pmet or the Pmetrika \LaTeX\ packages, they are \textit{not} required to do so.}

\comment{This document attempts to serve both (a) as an explanation of how to use Pmetrika \LaTeX2e\ Style File Package for Psychometrika Authors, and (b) as an example of its use.
If you are anxious to get started, please skip ahead to the
installation section (i.e., section \ref{install}).}

\thanks{In the spirit of David Thissen's 2001 Presidential Address to the
Psychometric Society, \textit{Psychometric Engineering as Art}, I
should acknowledge that this package could not have been created
without the \textit{prior art} of others. First there was the
original \texttt{psychometrika.cls} file created by Don DeLand of
Integre Technical Publishing, then the original draft of
\texttt{pmet.cls} created by Larry Hubert, and I couldn't begin to
list all the little pieces modeled from examples given in various
books expounding on all the multifarious nuances of \LaTeX. To all
those who have traveled this path before me, I give a nod and a
tip of my hat. I have served chiefly as an editor and synthesizer,
and I hope I have performed those tasks well.}

\contact{Correspondence should be sent to Tim Null, Technical
Editor, \textit{Psychometrika}, P.O. Box 7104, San Jose, CA
95150-7104 U.S.A.{\ } E-Mail: tim@timnull.com}

\end{titlepage}

\setcounter{page}{2}
\vspace*{2\baselineskip}

\RepeatTitle{HowTo Document: Pmetrika \LaTeX2e\ Style File Package for Psychometrika Authors}\vskip6pt

\abstracthead
%\centerline{\large{\bf Abstract}}

\begin{abstract}
Guidance on installing and using Version 2 of the
\textit{Psychometrika} \LaTeX2e\ Package for Authors.
\begin{keywords}
software installation, ``howto'' file, \LaTeX2e, templates,
\textit{Psychometrika}, IRT models, LISREL, Bayes estimates.
\end{keywords}
\end{abstract}\vspace{\fill}\pagebreak


\section{Introduction to Version 2 of the \LaTeX2e\ Package}

Welcome to Version 2 of the \textit{Psychometrika} \LaTeX2e\
Package for Authors. Hopefully you will find this package easy to
use, and you will find these instructions helpful. Should you have
any questions, suggestions, or problems, please contact me by
e-mail (tsnull@att.net). Although I'm \textit{not} a \TeX pert,
and I may not be able to immediately answer your question, fix
your problem, or implement your suggestion; I will be happy and
delighted to hear from you, and I will do what I can to help. User
input enables future refinement and improvement of this package,
so please don't be hesitant about contacting me.

\section{What's New in Version 2}

\begin{enumerate}

\item With the release of Version 2 we have instituted a new version
numbering system to help you quickly recognize when a package has
been revised. (Note that we will continue to use and update Version 1.)

\item Unlike Version 1, which replaced the normal \LaTeX2e\ article class file
(i.e.,~\texttt{article.cls}) with its own class file
(i.e.,~\texttt{pmet.cls}); Version 2 uses the normal \LaTeX2e\
article class file combined with two packages specifically
developed for \textit{Psychometrika} (i.e.,~\texttt{pmetrika.sty}
\&~\texttt{pmbib.sty}), and several other \LaTeX2e\
packages~(\texttt{fullpage.sty},
\texttt{ifthen.sty},~\texttt{indentfirst.sty},
\&~\texttt{graphicx.sty}). Template files, and all the required
packages are supplied with the package.

\end{enumerate}

\subsection{The New Version Numbering System}

Although we have now released Version 2 of the
\textit{Psychometrika} \LaTeX2e\ Package for Authors, we will
continue to offer Version 1. Both versions will be upgraded and
revised as the need arises. In addition, to make it easy to
recognize when a package has been modified, both versions will be
dated to indicate the date of the latest revision, and each
version number will be followed by a letter and a number, which
will be ascending as each version is updated. For example, on the
date this was written, the version number for Version 2 was 2A1.
The numbering sequence will then go from 2A1 through 2A9, then
from 2B1 through 2B9, and so forth. This should make it easy for
you to determine if the package that you use has been updated.
There will also be a~\texttt{readme.1st}~file that will outline
the revision history, and it will list the changes made in the
latest revision. By glancing at the revision date combined with
the version number, you should be able to quickly determine, if
the package you use has been updated; and by reading the
information in the~\texttt{readme.1st}~file, you should able to
decide if it is worth your time and trouble to download the latest
revisions\footnote{Unless problems are discovered in Versions 1 or
2, priority will be given to the items listed in section
\ref{plans}, rather than to additional revisions of Versions 1 and
2.}.

%\renewcommand{\baselinestretch}{1}
%\linespread{1}
\linespacing{1}
\begin{table}[h]
\vskip1pt\caption{\normalsize{Version 2 components}}\vskip3pt
\begin{tabular}{|l|l|}\hline
  \raisebox{-.5ex}{\textbf{File Name}} &   \raisebox{-.5ex}{\textbf{Purpose}} \\ \hline\hline
%  &    \\
\raisebox{-1ex}{{\tt article.tex}} & \raisebox{-1ex}{{\small Template for an article or note.}} \\[1ex]
  {\tt review.tex} & {\small Template for a book review.} \\[1ex]
  {\tt pmetrika.sty$^\dag$} & {\small Defines the appearance of the title page.} \\
   & {\small Redefines the section, subsection, subsubsection, and paragraph commands.} \\
   & {\small Redefines several environments, such as the theorem environment.} \\
   & {\small Sets ragged right margin and 1.5 line spacing.} \\
   & {\small Provides the end of proof command (i.e.,~\verb=\qed=) and symbol.}  \\
   & {\small Provides commands for bold Greek letters.} \\
   & \footnotesize{$^\dag$Requires~\texttt{indentfirst.sty}.} \\[1ex]
  {\tt pmbib.sty} & Redefines the reference section. (It can cause problems with Bib\TeX.)\\[1ex]
  {\tt fullpage.sty}\footnotesize$^{\ddag}$ & Defines~\verb=\textheight=~and~\verb=\textwidth=~based on the paper size. \\
  & {\footnotesize $^\ddag$\copyright\ 1994 P.W. Daly; see Kopka \& Daly, 1999, pp. 343--345. Requires~{\tt ifthen.sty}.}\\[1.5ex] \hline
\end{tabular}
\end{table}
%\renewcommand{\baselinestretch}{1.5}
%\linespread{1.5}\small\normalsize
\linespacing{1.5}

\subsection{Version 2 Components}

Table 1 lists the components that make up Version 2 of the
\textit{Psychometrika} \LaTeX2e\ Package for Authors. Unlike
Version 1, which used its own \LaTeX2e\ class file
(i.e.,~\texttt{pmet.cls}), Version 2 uses the regular \LaTeX2e\
article class file (i.e.,~\texttt{article.cls}), combined with
several~\verb=\usepackage=~files that alter or add to the regular
article class file. For example, the first line in a Version 1
document is

\begin{center}\verb=\documentclass{pmet}=;\end{center}

\noindent whereas, the first several lines in a typical Version 2
document could include the following commands:

%\linespread{1.2}\small\normalsize
\linespacing{1.2}
\begin{center}
\begin{tabular}{l}
\verb=\documentclass[titlepage,12pt]{article}=  \\
\verb=\usepackage[myheadings]{fullpage}=  \\
\verb=\usepackage{pmetrika}=  \\
\verb=\usepackage{pmbib}=  \\
\verb=\usepackage[dvips]{graphicx}=  \\[1.75ex]
\end{tabular}
\end{center}
%\linespread{1.5}\small\normalsize
\linespacing{1.5}

\noindent If you use A4 paper, the first line of your file should
be:\par

\begin{center}\verb=\documentclass[titlepage,a4paper,12pt]{article}=\end{center}

\noindent An explanation of what the above preamble commands mean,
and how they effect your \LaTeX2e\ output, can be found in the
section of using the provided template files (i.e., section
\ref{template}).

\shortpage

\subsection{Why We Made The Change}

Although only a few authors have indicated they have had problems
with Version 1 of our \LaTeX\ package, it has \textit{not} been
widely adopted by our authors who use \LaTeX; so in order to
promote the use of our \LaTeX2e\ package among authors, I wanted
Version 2 to have the following characteristics:
\begin{enumerate}
\item It should be easier to use than Version 1.
\item It should be as flexible, modifiable\footnote{If you modify any of
the files used in our author package, please rename the file, so
your new file won't be confused with the original.}, and
trouble-free as possible.
\item It should produce an attractive looking document appropriate for
submission to the Editor of \textit{Psychometrika}.
\end{enumerate}

\noindent It is my hope that these goals have been reached, but if
you have suggestions on how our \LaTeX2e\ packages can be
improved, please let me know (tsnull@att.net).

\subsection{Which Version is Right for You?}

You might be asking yourself, ``Which version is right for me?'' I
would suggest that the following people might want to use or
continue to use Version 1:

\begin{enumerate}

\item People who have used or attempted to use Version 1 in the past, and who were satisfied
with the results.

\item People who like the option of being able to print a manuscript in
a format that is quite similar to the final format that will
appear in the journal\footnote{Unfortunately this does require
some editing of the template file.}.

\end{enumerate}

\shortpage

The following people might want to try Version 2:

\begin{enumerate}

\item People who have used Version 1, and who either had
problems or who were dissatisfied with the results.

\item People who cannot resist the temptation of trying something new.

\item People who believe Version 2 generates a
more attractive manuscript than Version 1 (based on this
document or others they have seen).

\item If you have never used Version 1, I would recommend that you
use Version 2.

\end{enumerate}

\subsection{What Is Planned for the Future}\label{plans}

As has been mentioned, we currently offer versions 1 and 2 of our
author package for \LaTeX2e\ users. Eventually we hope to also
offer packages (or at least templates and ``howto'' files) for the
following users (possibly in the following order):

\begin{enumerate}
\item Templates for MS Word and WordPerfect users.
\item A package specifically designed to meet the special needs of Scientific Word/Scientific Workplace users.
\item Additional tweaking of the Table and Figure environments in Version 2 of the \LaTeX2e\ package.
\item Example files for the \LaTeX2e\ packages.
\item Develop and provide documentation for the LyX template Krishna
Tateneni provided.
\item A package that will work easily with \LaTeX\ 2.09.
\end{enumerate}

\section{Installation}\label{install}

Version 2 of the \LaTeX2e\ package assumes that the
following files are included in your \LaTeX\ installation:

%\linespread{1.2}\small\normalsize
\linespacing{1.2}
\begin{center}
\begin{tabular}{|l|l|}\hline
  \textbf{File Name} &   \textbf{Purpose} \\ \hline\hline
  \texttt{ifthen.sty} & Required by the fullpage package. \\
  \texttt{indentfirst.sty} & Required by the pmetrika package. \\
  \texttt{graphicx.sty} & Used to display figures. \\ \hline
\end{tabular}
\end{center}\vspace{0.5\baselineskip}
%\linespread{1.5}\small\normalsize
\linespacing{1.5}

These files
(i.e.,~\texttt{ifthen.sty},~\texttt{indentfirst.sty},~and~\texttt{graphicx.sty})~are
included with most \LaTeX\ installations, but just to make sure
you have them, I've included them in the~``\texttt{Other
Packages}''~subdirectory.

\subsection{Quick Installation}

To do a quick installation:

\begin{enumerate}

\item Copy the following files to your working
directory:\ \texttt{article.tex},\ \texttt{fullpage.sty}\
\texttt{pmetrika.sty},\ and\ \texttt{pmbib.sty}.

\item Rename the~\texttt{article.tex}~file (e.g., \texttt{ishmael.tex}).

\item Open the new file (i.e., the~\texttt{article.tex}~file that has been renamed) in your favorite \LaTeX\
editor, and compile it to make sure everything is working
(e.g.,~\texttt{latex ishmael}).

\item If all is well, then \textit{commence thy tour de force}.

\end{enumerate}

\subsection{A More ``Permanent'' Installation}

You will probably want to do the quick installation initially just
to see if Version 2 of the package works on your system. If you
make frequent contributions to \textit{Psychometika}, or you don't
like to ``clutter-up'' your working directory, you will probably
want to do the following installation, which will give you
access to the package files from any directory. (These directions
are applicable to MiKTeX Version 1.20e installed on
drive~\verb=C=~of a computer
running~\textsf{Microsoft\textsuperscript{\textregistered}~Windows},
but the general principles apply to other installations as well.)

%\longpage %The \longpage command will increase \textheight by 1 \baselineskip
\shortpage %The \longpage command will increase \textheight by 1 \baselineskip

\begin{enumerate}

\item In the \verb=c:\localtexmf= directory create a directory for the
package files; for example,~\verb=c:\localtexmf\tex\latex\pmetrika=.

\item Copy the following files to the directory created in Item
1:\ \texttt{fullpage.sty}\ \texttt{pmetrika.sty},\ and\
\texttt{pmbib.sty}.

\item Refresh the file name database for your installation of
\LaTeX. For example, the following command could be used with
MiKTeX 1.20e:
\begin{verbatim}C:\texmf\miktex\bin\initexmf.exe --mkpsres --search --update-fndb\end{verbatim}

\item Copy the~\texttt{article.tex}~template to your
working directory, and rename it.

\item Open the renamed file in your favorite \LaTeX\ editor, do a test compile to make
sure everything is working, then commence to \textit{Begin the
Beguine}.

\end{enumerate}

\section{Using the~\texttt{\bf article.tex}~Template}\label{template}

In the article template you will find ``comment'' lines
(i.e., lines which begin with one or more~\verb=%=~characters)
that direct you to seek further explanation in this document. For
example, the first such line states
\begin{center}\verb=%%ITEM 1 [See the ``howto.tex'' file]=\end{center}
A few lines later, you'll find
\begin{center}\verb=%%ITEM 2 [See the ``howto.tex'' file]=\end{center}
Throughout the template you'll find eleven such lines. Each item 1
through 11 is followed by one or more lines of \LaTeX\ code that
will need your attention. Explanations of the \LaTeX\ code and the
entries that are needed will follow in the next several pages.


\subsection{The Preamble}

\longpage

\noindent\texttt{\bf ITEM 1}\\[1.5ex]

If you use 8.5 by 11 inch paper begin your document with
\begin{center}\verb=\documentclass[titlepage,12pt]{article}=\end{center}
If you're using A4 paper begin with
\begin{center}\verb=\documentclass[titlepage,a4paper,12pt]{article}=\end{center}


\noindent\texttt{\bf ITEM 2}\\[1.5ex]

Feel free to use whatever \LaTeX\ graphics package you usually use
(e.g., epsf, graphics, graphicx) to include graphic files in your
\LaTeX2e\ document. I've set up the article template to use the
graphicx package, because that's the package our composition
company uses, and it's the one I'm most familiar with; but there's
no requirement that you use the graphicx package. I've included
four possible commands in the preamble of the template. Two are
reserved for people who use pdf\TeX. When you want your figures
included in your DVI or postscript file, you will normally use the
following command:

\begin{center}\verb=\usepackage[dvips]{graphicx}=\end{center}

\noindent When viewing a dvi file, it frequently takes a long time
for graphic files to be ``loaded'' into memory, so, while you're
in the process of writing your manuscript, you will probably want
to use the following command:

\begin{center}\verb=\usepackage[draft,dvips]{graphicx}=\end{center}

\noindent The ``draft'' option in the latter command will result
in an outline of a box the size of the graphic being shown in the
location where the graphic would otherwise be displayed. Your dvi
files should then load quickly in your dvi viewer, and this will
save a great deal of time, while you're working on the composition
of your document. (For more information regarding the use of the
graphicx package, and the inclusion of figures, see section
\ref{figs} below.)\\[1.5ex]

\noindent\texttt{\bf ITEM 3}\\[1.5ex]

If you've properly installed the packages (see section
\ref{install}), you probably shouldn't need to alter the following
lines:

%\linespread{1.2}\small\normalsize
\linespacing{1.2}
\begin{center}
\begin{tabular}{l}
\verb=\usepackage[myheadings]{fullpage}=  \\
\verb=\usepackage{pmetrika}=  \\
\verb=\usepackage{pmbib}=  \\[1.75ex]
\end{tabular}
\end{center}
%\linespread{1.5}\small\normalsize
\linespacing{1.5}

\noindent But if you use Bib\TeX, you may experience problems with
the pmbib package. If that's the case, ``comment out'' the~
\verb=\usepackage=~line calling the pmbib package with an \verb=%= symbol;
that is,

\begin{center}\verb=%\usepackage{pmbib}=\end{center}

\noindent Once that has been done, your reference section will
take on the characteristics that are specified in the normal
article class file. You may, or may not, find that result
aesthetically pleasing; but, if you're a Bib\TeX\ user, it's
better than giving
up your Bib\TeX\footnote{Someday I may figure out how to eliminate this incompatibility, and someday I may pay off my mortgage.}.\\[1.5ex]

\noindent\texttt{\bf ITEM 4}\\[1.5ex]

By ``default'' \textit{Psychometrika} articles do \textit{not}
have numbered sections, but authors are free to use numbered
sections, if they choose. If you wish to have numbered sections,
remove the~\verb=%=~symbol in front of the~\verb=\setcounter{secnumdepth}{3}=~command.\\[1.5ex]

\noindent\texttt{\bf ITEM 5}\\[1.5ex]

In this section enter your own~\verb=\usepackage=,~\verb=\input=,
and~\verb=\newcommand=~entries.

\subsection{The Title Page}\label{titlepage}

\noindent\texttt{\bf ITEM 6}\\[1.5ex]

The commands in this section combine to form your title page. Some
of the commands are pretty much self-evident; for example, your
title is entered between the braces in
the~\verb=\title{}=~command. All the letters in the title will be
made uppercase, so it doesn't matter how they are typed.

Type all of the author names between the braces in
the

\begin{center}\verb=\markright{\MakeLowercase{\textsc{}}}=\end{center}

\noindent command, such as is shown below

\begin{center}\verb=\markright{\MakeLowercase{\textsc{H. Melville and J.F. Cooper}}}=\end{center}

\noindent This will cause ``\MakeLowercase{\textsc{H. Melville and
J.F. Cooper}}'' to appear in the header of every page, unless you
enter a command to change the page style
(e.g.,~\verb=\thispagestyle{empty}=);
therefore, the header on page 33 would look like this:\\[1.5ex]

\noindent \MakeLowercase{\textsc{H. Melville and J.F. Cooper}}\hspace{\fill}33\\[1.5ex]

If a manuscript has only one author, then the author's name is
typed in the~\verb=\author{}=~field using upper and lowercase
(e.g.,~\verb=\author{Herman Melville}=); and the author
affiliation is entered in the~\verb=\affil{}=~field
(e.g.,~\verb=\affil{Pequod University}=). Use separate author and
affiliation entries, except in cases where two or more authors
have the same affiliation; for example, if the first and second
author have the same affiliation, but the third author has a
different affiliation, the first and second author names would be
entered in a single~\verb=\author{}=~followed by a
single~\verb=\affil{}=~field, and the third author's name and
affiliation would be entered in
separate~\verb=\author{}=~and~\verb=\affil{}=~fields. The names
and affiliation would appear on the title page like this:

\begin{center}
\textsc{Dorothy Parker and William Faulkner}\\

\MakeLowercase{\textsc{University of Algonquin}}\\

\vskip6pt

\textsc{James Fenimore Cooper}\\

\MakeLowercase{\textsc{Leatherstocking Institute}}\\

\end{center}

There are three additional commands unique to the title
page:~\verb=\comment=,~\verb=\thanks=, and~\verb=\contact=. All
three do the same thing; that is, they create an unnumbered
footnote near the bottom of the title page. You could use just one
of the commands several times, but I've created three commands
just to make it easy to conceptually organize the footnotes on the
title page. The ``comment'' command is for statements like:
\textit{This research was funded by my Aunt Josephine}. The
``thanks'' command is a modification of the
\verb=\thanks{}=~command already present in \LaTeX, and it is used
for statements like: \textit{I would very much like to thank my
Aunt Josephine}. The ``contact'' command gives you a place to list
things like your mailing and e-mail addresses. (These commands are
similar to the~\verb=\footnotetext{}=~command, in that they should
{\bf \textit{not}} be used within other commands, such
as~\verb=\author{}=~or~\verb=\title{}=.)

\subsection{The Abstract Page}\label{abstract}

\noindent\texttt{\bf ITEM 7}\\[1.5ex]

The~\verb=\abstracthead=~command gives the abstract it's heading.
You will need to enter your abstract in the space between
the~\verb=\begin{abstract}=, and
the~\verb=\begin{keywords}=~commands. Then enter your key words in the
space between the~\verb=\begin{keywords}=~and
the~\verb=\end{keywords}=~commands. Except for items like acronyms
and proper names, the key words should be in lowercase. (See the
key words in this manuscript for examples.)


\subsection{The Body of the Manuscript}\label{body}

\noindent\texttt{\bf ITEM 8}\\[1.5ex]\label{floathere}

The main body of your manuscript and appendices (if you have any)
go between the abstract and reference pages. Your Tables and
Figures should go after the reference section, but you will need
to mark the \textit{approximate} location where you wish your
tables and figures placed. (Since tables and figures are
``floats'' the actual location may be slightly before or after the
location you indicate.) Version 2 of our \LaTeX2e\ package
provides a ``tablehere'' and ``figurehere'' command to make it
very easy for you to mark the approximate location for your tables
and figures. (See the next paragraph for a description of these
commands.)

\paragraph{A couple new commands.} The pmetrika package defines new commands to indicate the
approximate location of your tables and figures. The
command~\verb=\tablehere{}=~can be used to mark the location of a
table. The command~\verb=\figurehere{}=~can be used to mark the
location of a figure. For example, the
command~\verb=\tablehere{1}=~would mark the location for Table 1
as shown below:

\tablehere{1}

\noindent The command~\verb=\figurehere{5}=~would mark the
location for Figure 5, as shown below:

\figurehere{5}%\vspace{\fill}%\pagebreak

\noindent \verb=\ref{}=~commands can also be used, rather than
entering a specific figure or table number; for example, in this
manuscript~\verb=\figurehere{\ref{textwidth}}=~would result in

\figurehere{\ref{textwidth}}

For additional information about figures, see section \ref{figs}.

\paragraph{Section headings.} In \textit{Psychometrika} there are three types of main headings used, and a
paragraph heading. Examples are given below:

\section*{This is a Section Heading}

Section headings are centered, Roman\footnote{Although I've chosen
to use boldface for the \LaTeX2e\ section headings, in the journal
they are Roman (i.e., normalfont, not italic or bold).}, with long
and major words capitalized.

\subsection*{This is a Subsection Heading}

Subsection headings are centered, italic, with long and major
words capitalized.

\subsubsection*{This is a Subsubsection Heading}

Subsubsection headings are flushleft, italic, with long and major
words capitalized.

\paragraph{This is a paragraph heading.} Paragraph headings are composed of
a single word or a short phrase. The word or phrase is followed
by a period. Paragraph headings are italic with only the first letter of the
first word capitalized (except for acronyms and proper names,
of course).\\[1.5ex]

\noindent\texttt{\bf ITEM 9}\\[1.5ex]

\paragraph{Appendices.} Item 9 is followed by a number of
commands, which---if you have one or more appendices---you can
``activate'' by removing the~\verb=%=~at the front of the line.
If you only have one appendix, you should remove
the~\verb=%=~character in front of the following lines:

\begin{center}
%\linespread{1.2}\small\normalsize
\linespacing{1.2}
\begin{tabular}{l}
\verb=\appendix= \\
\verb=\renewcommand{\theequation}{A\arabic{equation}}=  \\
\verb=\setcounter{equation}{0}=  \\
\verb=\renewcommand{\thesection}{\Alph{subsection}}= \\
\verb=\setcounter{section}{0}= \\
\verb=\section*{Appendix}= \\
\verb=%\section*{Appendix A}= \\
\verb=%\section*{Appendix B}= \\
\end{tabular}
\end{center}
%\linespread{1.5}\small\normalsize
\linespacing{1.5}

\noindent If you only have two appendices, you should remove
the~\verb=%=~character in front of the following lines:


\begin{center}
%\linespread{1.2}\small\normalsize
\linespacing{1.2}
\begin{tabular}{l}
\verb=\appendix= \\
\verb=\renewcommand{\theequation}{A\arabic{equation}}=  \\
\verb=\setcounter{equation}{0}=  \\
\verb=\renewcommand{\thesection}{\Alph{subsection}}= \\
\verb=\setcounter{section}{0}= \\
\verb=%\section*{Appendix}= \\
\verb=\section*{Appendix A}= \\
\verb=\section*{Appendix B}= \\
\end{tabular}
\end{center}
%\linespread{1.5}\small\normalsize
\linespacing{1.5}

\noindent (Of course the text for Appendix A and Appendix B would
follow the respective headings.) If then you added a third and
forth appendix, they would have the following section headings:

\hspace{6em}\verb=\section*{Appendix C}=  \\[1ex]
\hspace{6em}\verb=\section*{Appendix D}= \\


\subsection{The Reference Section}

\noindent\texttt{\bf ITEM 10}\\[1.5ex]

You shouldn't have to do anything special or different in the
reference section. Just start each item with either
\verb*=\bibitem = or \verb*=\item = (where the ``\verb*= =''
symbol stands for a blank space), and then enter your
bibliographic information as you would normally. The pmbib package
should be able to format the lines correctly for you. (If you have
a problem with pmbib, I may be able to help; unless, of course,
the problem is due to a conflict with Bib\TeX.) If you have
stylistic questions, consult the \textit{APA Publication Manual}.

\subsection{Figures and Tables}\label{figs}

\noindent\texttt{\bf ITEM 11}\\[1.5ex]

When it comes to things like figures and tables, I find examples
more helpful than 10,000 words, so I encourage you to check out
the examples that follow the reference section.

The pmetrika package defines a new command for spacing between
figures (i.e.,~\verb=\figskip=).

\section{New Commands}

\subsection{End of Proof Symbol (\/$\backslash$qed)}\label{qed}

\paragraph{The bad news.} The ``qed'' command may be
incompatible with the ``subequation'' and ``subeqnarry'' packages;
and if you use the ``fleqn'' package, fleqn must be loaded after
the pmetrika package. Also, the~\verb=\qed=~command cannot be used
within \verb=\[...\]=~environments.

\paragraph{The good news.} The~\verb=\qed=~command provides the end of proof symbol. To
see how it works, please review the following
\textit{spoof-proof}:

\begin{proof}
It's obvious.\qed
\end{proof}

\subsection{Boldface Greek Letters}

\linespacing{1.2}
\begin{longtable}{l@{\qquad}l}
\verb=$\bfEpsilon$= &  No Varpi in Computer Modern Bold  \kill
\textbf{Command} & \textbf{Result} \endhead\hline
%Begin uppercase
\verb=$\bfAlpha$= & $\bfAlpha$ \\
\verb=$\bfBeta$= & $\bfBeta$ \\
\verb=$\bfPsi$= & $\bfPsi$ \\
\verb=$\bfDelta$= & $\bfDelta$ \\
\verb=$\bfEpsilon$= & $\bfEpsilon$ \\
\verb=$\bfPhi$= & $\bfPhi$ \\
\verb=$\bfGamma$= & $\bfGamma$ \\
\verb=$\bfEta$= & $\bfEta$ \\
\verb=$\bfIota$= & $\bfIota$ \\
\verb=$\bfXi$= & $\bfXi$ \\
\verb=$\bfKappa$= & $\bfKappa$ \\
\verb=$\bfLambda$= & $\bfLambda$ \\
\verb=$\bfMu$= & $\bfMu$ \\
\verb=$\bfNu$= & $\bfNu$ \\
\verb=$\bfPi$= & $\bfPi$ \\
\verb=$\bfTheta$= & $\bfTheta$ \\
\verb=$\bfRho$= & $\bfRho$ \\
\verb=$\bfSigma$= & $\bfSigma$ \\ \hline
\verb=$\bfTau$= & $\bfTau$ \\
\verb=$\bfVartheta$= & $\bfVartheta$ \\
\verb=$\bfOmega$= & $\bfOmega$ \\
\verb=$\bfVarpi$= & No Varpi in Computer Modern Bold \\
\verb=$\bfUpsilon$= & $\bfUpsilon$ \\
\verb=$\bfZeta$= & $\bfZeta$ \\
%End uppercase
%Begin lowercase
\verb=$\bfalpha$= & $\bfalpha$ \\
\verb=$\bfbeta$= & $\bfbeta$ \\
\verb=$\bfchi$= & $\bfchi$ \\
\verb=$\bfpsi$= & $\bfpsi$ \\
\verb=$\bfdelta$= & $\bfdelta$ \\
\verb=$\bfepsilon$= & $\bfepsilon$ \\
\verb=$\bfphi$= & $\bfphi$ \\
\verb=$\bfgamma$= & $\bfgamma$ \\
\verb=$\bfeta$= & $\bfeta$ \\
\verb=$\bfiota$= & $\bfiota$ \\
\verb=$\bfxi$= & $\bfxi$ \\
\verb=$\bfkappa$= & $\bfkappa$ \\
\verb=$\bflambda$= & $\bflambda$ \\
\verb=$\bfmu$= & $\bfmu$ \\
\verb=$\bfnu$= & $\bfnu$ \\
\verb=$\bfpi$= & $\bfpi$ \\
\verb=$\bftheta$= & $\bftheta$ \\
\verb=$\bfrho$= & $\bfrho$ \\
\verb=$\bfsigma$= & $\bfsigma$ \\
\verb=$\bftau$= & $\bftau$ \\
\verb=$\bfvartheta$= & $\bfvartheta$ \\
\verb=$\bfomega$= & $\bfomega$ \\
\verb=$\bfvarpi$= & $\varpi$ \\
\verb=$\bfvarphi$= & $\varphi$ \\
\verb=$\bfupsilon$= & $\upsilon$ \\
\verb=$\bfzeta$= & $\bfzeta$ \\ \hline
%End lowercase
\end{longtable}
\linespacing{1.5}

\vskip6pt

For other boldface characters in math mode, please use the
$\mathbf{mathbf}$ command (i.e.,~\verb=\mathbf{}=) rather than the
$\boldmath{boldmath}$ command (i.e.,~\verb=\boldmath{}=), so that
you don't end up combining \textit{italic} and \textbf{boldface}
like {\bf \textit{this}}.

\subsection{Miscellaneous New Commands}

Below I have listed the new commands created by the pmetrika
package in the table shown below\footnote{Please note that the
following ``newtheorems'' have already been defined in the
pmetrika package: assumption, axiom, corollary, definition,
example, exercise, lemma, remark, proposition, and theorem.}.

\linespacing{1.2}
\begin{longtable}{l@{\qquad}l}
 \verb=\shortpage= & Decrease \verb=\textheight= by 1 \verb=\baselineskip=. \kill
\textbf{Command} & \textbf{Result} \endhead \hline
% \verb== &  \\
 \verb=\abstracthead= & see section \ref{abstract} \\
 \verb=\comment= & see section \ref{titlepage}  \\
 \verb=\contact= & see section \ref{titlepage}  \\
 \verb=\figskip= & Vertical space between figures.  \\
 \verb=\figurehere= & see section \ref{floathere}  \\
 \verb=\linespacing{}= & Change baselinestretch to specified number  \\
 \verb=\longpage= & Increase \verb=\textheight= by 1 \verb=\baselineskip=.  \\
 \verb=\qed= & see section \ref{qed}  \\
 \verb=\shortpage= & Decrease \verb=\textheight= by 1 \verb=\baselineskip=.  \\
 \verb=\tablehere= & see section \ref{floathere}  \\
 \verb=\thanks= & see section \ref{titlepage}  \\ \hline
\end{longtable}
\linespacing{1.5}


\vspace{\fill}\pagebreak

\begin{thebibliography}

\bibitem Goossens, Michel, Mittelbach, Frank, \& Samarin,
Alexander. (1994). \textit{The \LaTeX\ companion} (2nd ed.).
Reading, MA: Addison-Wesley.

\bibitem Griffiths, David F., \& Higham, Desmond J. (1997).
\textit{Learning \LaTeX}. Philadelphia, PA: Society for Industrial
and Applied Mathematics.

\bibitem Kopka, Helmut, \& Daly, Patrick W. (1999). \textit{A guide to
\LaTeX\ document preparation for beginners and advanced users}
(3rd ed.). Harlow, England: Addison-Wesley.

\bibitem Oetiker, Tobias, Partl, Hubert, Hyna, Irene, \& Schlegl,
Elisabeth. (2000). \textit{The not so short introduction to
\LaTeX2e: Or \LaTeX2e\ in 95 minutes} (Version 3.20). Cambridge,
MA: Free Software Foundation.
\mbox(http://www.ctan.org/tex-archive/info/lshort/lshort.pdf)

\bibitem \textit{Publication manual of the American Psychological
Association} (5th ed.). (2001). Washington, DC: APA.

\bibitem Reckdahl, Keith. (1997). \textit{Using imported graphics in
\LaTeX2e}. Cambridge, MA: Free Software Foundation.
\mbox(http://www.ctan.org/tex-archive/info/epslatex.pdf)

\bibitem Thissen, David. (2001). Psychometric engineering as art.
\textit{Psychometrika}, \textit{66}, 473--485.

\end{thebibliography}


\linespacing{1.1}\vspace{\fill}\begin{quote}\textbf{\textit{Note: Include your figures and tables after the reference section, but be sure to mark in your paper the approximate location where you want your tables and figures to be placed.}}\end{quote}\vspace{\fill}\pagebreak
%%If you choose to integrate your figures and tables into the text of your paper, be sure to provide camera-ready copies of your figures on separate sheets of paper.
%\renewcommand{\baselinestretch}{1}\small\normalsize
%\linespread{1}\small\normalsize
\linespacing{1}
\flushbottom


%\section*{Tables}\setcounter{table}{0}

%%PUT YOUR FIGURES AND TABLES AFTER THE REFERENCE SECTION

\section*{Figures}

\vspace{\fill}

\begin{figure}[hp]
\begin{center}
\includegraphics{p-logo.eps}
%\includegraphics{figs/p-logo.eps}
%\includegraphics{figs/p-logo.pdf}
\caption{The \textit{Psychometrika} logo. This is the same as the
scale being equal to 1.}
\end{center}
\end{figure}

\figskip

\begin{figure}[hp]
\begin{center}
\includegraphics[width={\textwidth}]{p-logo.eps}
%\includegraphics[width={\textwidth}]{figs/p-logo.pdf}
\caption{The \textit{Psychometrika} logo with a width set to equal
to the width of the page.} \label{textwidth}
\end{center}
\end{figure}

\figskip

\begin{figure}[hp]
\begin{center}
\includegraphics[scale=.5]{p-logo.eps}
%\includegraphics[scale=.5]{figs/p-logo.pdf}
\caption{The \textit{Psychometrika} logo with scale set equal to
0.5.} \label{scale}
\end{center}
\end{figure}



\begin{figure}[hp]
\begin{center}
\rotatebox{90}{\includegraphics[width=2.5in]{p-logo.eps}}
%\rotatebox{90}{\includegraphics[width=2.5in]{figs/p-logo.pdf}}
\caption{The \textit{Psychometrika} logo rotated 90 degrees with
width equal to 2.5 inches.} \label{rotate}
\end{center}
\end{figure}

\figskip

\begin{figure}[hp]
\begin{center}
\rotatebox{90}{\includegraphics[scale=.2]{p-logo.eps}}
%\rotatebox{90}{\includegraphics[scale=.2]{figs/p-logo.pdf}}
\caption{The \textit{Psychometrika} logo rotated with scale set
equal to 0.2.} \label{scale.rotate}
\end{center}
\end{figure}

% The following figures have both the graphic and the caption rotated.

\begin{figure}[t]
\begin{center}
\rotatebox{90}{\begin{minipage}{2in}
\includegraphics[width=2in]{p-logo.eps}
%\includegraphics[width=2in]{figs/p-logo.pdf}
\caption{A figure with both the graphic and caption rotated.}
\end{minipage}}
\end{center}
\end{figure}

\figskip

\begin{figure}[b]
\centerline{\rotatebox{90}{\begin{minipage}{5in}
\includegraphics[width=5in]{p-logo.eps}
%\includegraphics[width=5in]{figs/p-logo.pdf}
\caption{A figure with both the graphic and caption rotated.}
\end{minipage}}}
\end{figure}


\begin{figure}[b]
\centerline{\rotatebox{90}{\begin{minipage}{\textheight}
\includegraphics[width={\textheight}]{p-logo.eps}
%\includegraphics[width={\textheight}]{figs/p-logo.pdf}
\caption{A figure with both the graphic and caption rotated.}
\end{minipage}}}
\end{figure}


\end{document}
