%%This article template was last updated 30 September 2003 --- Tim Null
%%It is part of the
%%Pmetrika LaTeX Style File Package for Psychometrika Authors
%%This is the \LaTeX2e "article" template for the journal Psychometrika.
%%It can be freely used and modified by anyone.
%%
%% ITEM 1 [See the "howto.tex" file.]
%%
\documentclass[titlepage,11pt]{article}
%\documentclass[titlepage,a4paper,11pt]{article}

%%ITEM 2 [See the "howto.tex" file.]
\usepackage[dvips]{graphicx}
%\usepackage[draft,dvips]{graphicx}
%\usepackage[pdftex]{graphicx}
%\usepackage[draft,pdftex]{graphicx}

%% ITEM 3 [See the "howto.tex" file.]
\usepackage[myheadings]{fullpage}
\usepackage{pmetrika}
\usepackage{pmbib}

\usepackage{submit}

%% ITEM 4 [See the "howto.tex" file.]
%\setcounter{secnumdepth}{3}


%% ITEM 5 [See the "howto.tex" file.]
%%%%%%%%%%%%%%%%%%%%%%%%%%%%%%%%%%%%%%%%%%%%%%%%%%%%%%%
%%   ENTER YOUR PERSONAL PREAMBLE ITEMS IN THIS SECTION
%%             Some examples provided
%%%%%%%%%%%%%%%%%%%%%%%%%%%%%%%%%%%%%%%%%%%%%%%%%%%%%%%
%%  Begin Section  %%%%%%%%%%%%%%%%%%%%%%%%%%%%%%%%%%%%
%%%%%%%%%%%%%%%%%%%%%%%%%%%%%%%%%%%%%%%%%%%%%%%%%%%%%%%
%\usepackage{}
%\usepackage{}
%input
\def\diag{\mathrm{diag}}
\def\vec{\mathrm{vec}}
\def\cov{\mathrm{cov}}
\def\tr{\mathrm{Tr}}
\def\pr{\mathrm{Pr}}
%%%%%%%%%%%%%%%%%%%%%%%%%%%%%%%%%%%%%%%%%%%%%%%%%%%%%%%
%%  End Section  %%%%%%%%%%%%%%%%%%%%%%%%%%%%%%%%%%%%%%
%%%%%%%%%%%%%%%%%%%%%%%%%%%%%%%%%%%%%%%%%%%%%%%%%%%%%%%



\begin{document}

%% ITEM 6 [See the "howto.tex" file.]
\begin{titlepage}

\title{}

%%Disable \markright for your submission,
%%if it includes any author names.
%%Note: The "submit" package includes a running header.
%\markright{\MakeLowercase{\textsc{}}}

\author{}

\affil{}

%\author{}
%\affil{}

\vspace{\fill}\centerline{\today}\vspace{\fill}

%\comment{This research was funded by .}
%\thanks{I would like to thank .}
\linespacing{1}
\contact{Correspondence should be sent to\\

\noindent E-Mail: \break
\noindent Phone: \break
\noindent Fax: \break
\noindent Website:  }

\end{titlepage}

%% ITEM 7 [See the "howto.tex" file.]
\setcounter{page}{2}
\vspace*{2\baselineskip}

\RepeatTitle{Your Title Goes Here Again}\vskip3pt

\linespacing{1.5}
%% ITEM 8 [See the "howto.tex" file.]
\abstracthead
\begin{abstract}


\begin{keywords}

\end{keywords}
\end{abstract}\vspace{\fill}\pagebreak

%% ITEM 8 [See the "howto.tex" file.]
\section{Introduction}


\vspace{\fill}\pagebreak

%% ITEM 9 [See the "howto.tex" file.]
%\appendix
%\renewcommand{\theequation}{A\arabic{equation}}
%\setcounter{equation}{0}
%\renewcommand{\thesection}{\Alph{subsection}}
%\setcounter{section}{0}
%\section*{Appendix}
%\section*{Appendix A}
%\section*{Appendix B}
%\vspace{\fill}\pagebreak

%% ITEM 10 [See the "howto.tex" file.]
\begin{thebibliography}

\bibitem

\end{thebibliography}

%% ITEM 11 [See the "howto.tex" file.]
%%%% You can put your Figures and Tables here
%%%% after the Reference Section.
%%%% BE SURE TO MARK IN THE TEXT WHERE
%%%% YOU WANT EACH FIGURE AND TABLE TO BE PLACED.
%%%% If you prefer, you can integrate your figures and tables into the text of your paper,
%%%% PROVIDED you will provide camera-ready copies of each figure.
%\vspace{\fill}\pagebreak
%\linespacing{1}

%\section*{Figures}
%
%\begin{figure}[h]
%\centerline{\includegraphics{figure01.eps}}
%\caption{Your figure caption goes here.}
%\end{figure}
%\vskip6pt


%\vspace{\fill}\pagebreak

%\section*{Tables}

%\vspace{\fill}\pagebreak

\end{document}
