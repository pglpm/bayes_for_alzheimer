%%%%%%%%%%%%%%%%%%%%%%%%%%%%%%%%%%%%%%%%%%%%%%%%%%%%%%%%%%%%%%%%%%%%%%%%%%%%%%%%%%%%%%%%%%%%%%%%%%%%%%%%%%%%%%%%%%%%%%%%%%%%%%%%%%%%%%%%%%%%%%%%%%%%%%%%%%%
% This is just an example/guide for you to refer to when submitting manuscripts to Frontiers, it is not mandatory to use Frontiers .cls files nor frontiers.tex  %
% This will only generate the Manuscript, the final article will be typeset by Frontiers after acceptance.   
%                                              %
%                                                                                                                                                         %
% When submitting your files, remember to upload this *tex file, the pdf generated with it, the *bib file (if bibliography is not within the *tex) and all the figures.
%%%%%%%%%%%%%%%%%%%%%%%%%%%%%%%%%%%%%%%%%%%%%%%%%%%%%%%%%%%%%%%%%%%%%%%%%%%%%%%%%%%%%%%%%%%%%%%%%%%%%%%%%%%%%%%%%%%%%%%%%%%%%%%%%%%%%%%%%%%%%%%%%%%%%%%%%%%

%%% Version 3.4 Generated 2022/06/14 %%%
%%% You will need to have the following packages installed: datetime, fmtcount, etoolbox, fcprefix, which are normally inlcuded in WinEdt. %%%
%%% In http://www.ctan.org/ you can find the packages and how to install them, if necessary. %%%
%%%  NB logo1.jpg is required in the path in order to correctly compile front page header %%%

\documentclass[utf8]{FrontiersinHarvard} % for articles in journals using the Harvard Referencing Style (Author-Date), for Frontiers Reference Styles by Journal: https://zendesk.frontiersin.org/hc/en-us/articles/360017860337-Frontiers-Reference-Styles-by-Journal
%\documentclass[utf8]{FrontiersinVancouver} % for articles in journals using the Vancouver Reference Style (Numbered), for Frontiers Reference Styles by Journal: https://zendesk.frontiersin.org/hc/en-us/articles/360017860337-Frontiers-Reference-Styles-by-Journal
%\documentclass[utf8]{frontiersinFPHY_FAMS} % Vancouver Reference Style (Numbered) for articles in the journals "Frontiers in Physics" and "Frontiers in Applied Mathematics and Statistics" 

%\setcitestyle{square} % for articles in the journals "Frontiers in Physics" and "Frontiers in Applied Mathematics and Statistics" 
\usepackage{url,hyperref,lineno,microtype,subcaption}
\usepackage[onehalfspacing]{setspace}
\usepackage[usenames]{xcolor}
% Tol (2012) colour-blind-, print-, screen-friendly colours, alternative scheme; Munsell terminology
\definecolor{bluepurple}{RGB}{68,119,170}
\definecolor{blue}{RGB}{102,204,238}
\definecolor{green}{RGB}{34,136,51}
\definecolor{yellow}{RGB}{204,187,68}
\definecolor{red}{RGB}{238,102,119}
\definecolor{redpurple}{RGB}{170,51,119}
\definecolor{grey}{RGB}{187,187,187}
\definecolor{lgrey}{RGB}{221,221,221}

\definecolor{notecolour}{RGB}{68,170,153}
%\newcommand*{\puzzle}{\maltese}
\newcommand*{\puzzle}{{\fontencoding{U}\fontfamily{fontawesometwo}\selectfont\symbol{225}}}
\newcommand*{\wrench}{{\fontencoding{U}\fontfamily{fontawesomethree}\selectfont\symbol{114}}}
\newcommand*{\pencil}{{\fontencoding{U}\fontfamily{fontawesometwo}\selectfont\symbol{210}}}
\newcommand{\mynotew}[1]{{\color{notecolour}\wrench\ #1}}
\newcommand{\mynotep}[1]{{\color{notecolour}\pencil\ #1}}
\newcommand{\mynotez}[1]{{\color{notecolour}\puzzle\ #1}}

\usepackage{wrapfig}



\providecommand{\href}[2]{#2}
\providecommand{\eprint}[2]{\texttt{\href{#1}{#2}}}
\newcommand*{\amp}{\&}
% \newcommand*{\citein}[2][]{\textnormal{\textcite[#1]{#2}}%\addtocategory{extras}{#2}
% }
\newcommand*{\citein}[2][]{\textnormal{\cite[#1]{#2}}%\addtocategory{extras}{#2}
}
\newcommand*{\citebi}[2][]{\cite[#1]{#2}%\addtocategory{extras}{#2}
}
\newcommand*{\subtitleproc}[1]{}
\newcommand*{\chapb}{ch.}
%
%\def\UrlOrds{\do\*\do\-\do\~\do\'\do\"\do\-}%
\def\myUrlOrds{\do\0\do\1\do\2\do\3\do\4\do\5\do\6\do\7\do\8\do\9\do\a\do\b\do\c\do\d\do\e\do\f\do\g\do\h\do\i\do\j\do\k\do\l\do\m\do\n\do\o\do\p\do\q\do\r\do\s\do\t\do\u\do\v\do\w\do\x\do\y\do\z\do\A\do\B\do\C\do\D\do\E\do\F\do\G\do\H\do\I\do\J\do\K\do\L\do\M\do\N\do\O\do\P\do\Q\do\R\do\S\do\T\do\U\do\V\do\W\do\X\do\Y\do\Z}%
\makeatletter
%\g@addto@macro\UrlSpecials{\do={\newline}}
\g@addto@macro{\UrlBreaks}{\myUrlOrds}
\makeatother
\newcommand*{\arxiveprint}[1]{%
arXiv \doi{10.48550/arXiv.#1}%
}
\newcommand*{\mparceprint}[1]{%
\href{http://www.ma.utexas.edu/mp_arc-bin/mpa?yn=#1}{mp\_arc:\allowbreak\nolinkurl{#1}}%
}
\newcommand*{\haleprint}[1]{%
\href{https://hal.archives-ouvertes.fr/#1}{\textsc{hal}:\allowbreak\nolinkurl{#1}}%
}
\newcommand*{\philscieprint}[1]{%
\href{http://philsci-archive.pitt.edu/archive/#1}{PhilSci:\allowbreak\nolinkurl{#1}}%
}
\newcommand*{\doi}[1]{%
\href{https://doi.org/#1}{\textsc{doi}:\allowbreak\nolinkurl{#1}}%
}
\newcommand*{\biorxiveprint}[1]{%
bioRxiv \doi{10.1101/#1}%
}
\newcommand*{\osfeprint}[1]{%
Open Science Framework \doi{10.31219/osf.io/#1}%
}
%% symbol = for equality statements within probabilities
\newcommand*{\mo}[1][=]{\mathord{\,#1\,}}
%%
\newcommand*{\sect}{\S}% Sect.~
\newcommand*{\sects}{\S\S}% Sect.~
\newcommand*{\chap}{ch.}%
\newcommand*{\chaps}{chs}%
\newcommand*{\bref}{ref.}%
\newcommand*{\brefs}{refs}%
%\newcommand*{\fn}{fn}%
\newcommand*{\eqn}{eq.}%
\newcommand*{\eqns}{eqs}%
\newcommand*{\fig}{fig.}%
\newcommand*{\figs}{figs}%
\newcommand*{\vs}{{vs}}
\newcommand*{\eg}{{e.g.}}
\newcommand*{\etc}{{etc.}}
\newcommand*{\ie}{{i.e.}}
%\newcommand*{\ca}{{c.}}
\newcommand*{\foll}{{ff.}}
%\newcommand*{\viz}{{viz}}
\newcommand*{\cf}{{cf.}}
%\newcommand*{\Cf}{{Cf.}}
%\newcommand*{\vd}{{v.}}
\newcommand*{\etal}{{et al.}}

%\usepackage{fancybox}
\usepackage{framed}


% \newenvironment{description}{}{}
% \usepackage[shortlabels,inline]{enumitem}
% \SetEnumitemKey{para}{itemindent=\parindent,leftmargin=0pt,listparindent=\parindent,parsep=0pt,itemsep=\topsep}
% \setlist{itemsep=0pt,topsep=\parsep}
% \setlist[enumerate,2]{label=(\roman*)}
% \setlist[enumerate]{label=(\alph*),leftmargin=1.5\parindent}
% \setlist[itemize]{leftmargin=1.5\parindent}
% \setlist[description]{leftmargin=1.5\parindent}


\usepackage{mathtools}

\usepackage[main=british]{babel}\selectlanguage{british}
%\newcommand*{\langnohyph}{\foreignlanguage{nohyphenation}}
\newcommand{\langnohyph}[1]{\begin{hyphenrules}{nohyphenation}#1\end{hyphenrules}}

\usepackage[autostyle=false,autopunct=false,english=british]{csquotes}
\setquotestyle{american}
\newcommand*{\defquote}[1]{`\,#1\,'}

\usepackage{upgreek}
%% Macros
\DeclarePairedDelimiter\abs{\lvert}{\rvert}
\DeclarePairedDelimiter\set{\{}{\}} %}
\newcommand*{\p}{\mathrm{p}}%probability
\renewcommand*{\P}{\mathrm{P}}%probability
\newcommand*{\E}{\mathrm{E}}
%% The "\:" space is chosen to correctly separate inner binary and external relationss
\renewcommand*{\|}[1][]{\nonscript\:#1\vert\nonscript\:\mathopen{}}
\newcommand*{\defd}{\coloneqq}
\newcommand*{\defs}{\eqqcolon}
\newcommand*{\Land}{\bigwedge}
\newcommand*{\zerob}[1]{\makebox[0pt][c]{#1}}
\newcommand*{\delt}{\updelta}
% 
\newcommand*{\ad}{Alzheimer's Disease}
\newcommand*{\mci}{Mild Cognitive Impairment}


% Leave a blank line between paragraphs instead of using \\

%\linenumbers


\def\keyFont{\fontsize{8}{11}\helveticabold }
\def\firstAuthorLast{Sample {et~al.}} %use et al only if is more than 1 author
\def\Authors{P.G.L. Porta~Mana\,$^{1,2,*}$, I.~Rye\,$^{3}$, A.~Vik\,$^{1,2}$, M.~Koci\'nski\,$^{2,4}$, A.~Lundervold\,$^{2,4}$, A.~J.~Lundervold\,$^{3}$, A.~S.~Lundervold\,$^{1,2}$}
% Affiliations should be keyed to the author's name with superscript numbers and be listed as follows: Laboratory, Institute, Department, Organization, City, State abbreviation (USA, Canada, Australia), and Country (without detailed address information such as city zip codes or street names).
% If one of the authors has a change of address, list the new address below the correspondence details using a superscript symbol and use the same symbol to indicate the author in the author list.
\def\Address{$^{1}$Department of Computer Science, Electrical Engineering and Mathematical Sciences, Western Norway University of Applied Sciences, Bergen, Norway \\
$^{2}$Mohn Medical Imaging and Visualization Centre (MMIV), Department of Radiology, Haukeland University Hospital, Bergen, Norway\\
$^{3}$Department of Biological and Medical Psychology, University of Bergen, Norway\\
$^{4}$Department of Biomedicine, University of Bergen, Norway}
% The Corresponding Author should be marked with an asterisk
% Provide the exact contact address (this time including street name and city zip code) and email of the corresponding author
\def\corrAuthor{P.G.L Porta~Mana, HVL, Inndalsveien 28, 5063 Bergen}
\def\corrEmail{pgl@portamana.org}




\begin{document}
\onecolumn
\firstpage{1}

%\title[Conversion from MCI to AD]{Conversion from MCI to Alzheimer's disease: model-free predictions with quantified uncertainty} 
%\title[Conversion from MCI to AD]{Model-free predictions with quantified uncertainty in personalized medicine: A case study on the conversion from MCI to AD} 
\title[Conversion from Mild Cognitive Impairment to Alzheimer's Disease]{Personalized prognosis \& decision:\\ An example study on the conversion from Mild Cognitive Impairment to Alzheimer's Disease} 

\author[\firstAuthorLast ]{\Authors} %This field will be automatically populated
\address{} %This field will be automatically populated
\correspondance{} %This field will be automatically populated

\extraAuth{}% If there are more than 1 corresponding author, comment this line and uncomment the next one.
%\extraAuth{corresponding Author2 \\ Laboratory X2, Institute X2, Department X2, Organization X2, Street X2, City X2 , State XX2 (only USA, Canada and Australia), Zip Code2, X2 Country X2, email2@uni2.edu}


\maketitle


\begin{abstract}

%%% Leave the Abstract empty if your article does not require one, please see the Summary Table for full details.
\section{}

Patients with Mild Cognitive Impairment have an increased risk of a trajectory toward Alzheimer's Disease. Early identification of patients with a high risk of Alzheimer's Disease is essential to provide treatment before the disease is well-established in the brain. great importance to study how well different kinds of predictors
%-- from neuropsychological examinations to advanced brain-imaging techniques -- 
allow us to prognose a trajectory from Mild Cognitive Impairment towards Alzheimer's Disease in an individual patient.

But more is needed for a personalized approach to prognosis, prevention, and treatment, than just the obvious requirement that prognoses be as best as they can be for each patient. Several situational elements that can be different from patient to patient must be accounted for:
\begin{itemize}
\item the \emph{kinds} of clinical data and evidence available for prognosis;
\item the \emph{outcomes} of the same kind of clinical data and evidence;
\item the kinds of treatment or prevention strategies available, owing to different additional medical factors such as physical disabilities, different attitudes toward life, different family networks and possibilities of familial support, different economic means;
\item the advantages and disadvantages, benefits and costs of the same kinds of treatment or prevention strategies; the patient has a major role in the quantification of such benefits and costs;
\item finally, the initial evaluation by the clinician -- which often relies on too subtle clues (family history, regional history, previous case experience) to be considered as measurable data.
\end{itemize}
Statistical decision theory is the normative quantification framework that takes into account these fundamental differences. Medicine has the distinction of having been one of the first fields to adopt this framework, exemplified in brilliant old and new textbooks on clinical decision-making.

Clinical decision-making makes allowance for these differences among patients through two requirements. First, the quantification of prognostic evidence on one side, and of benefits and costs of treatments and prevention strategies on the other, must be clearly separated and handled in a modular way. Two patients can have the same prognostic evidence and yet very different prevention options. Second, the quantification of independent prognostic evidence ought to be in the form of \emph{likelihoods about the health condition} (or equivalently of likelihood ratios, in a binary case), that is, of the probabilities of the observed test outcomes given the hypothesized health conditions. Likelihoods from independent clinical tests and predictors can then be combined with a simple multiplication; for one patient, we could have three kinds of predictor available; for another, we could have five. The clinician's pre-test assessment is included in the form of a probability. These patient-dependent probabilities are combined with the patient-dependent costs and benefits of treatment or prevention to arrive at the best course of action for that patient. The main result underlying statistical decision theory is that decision-making \emph{must} take this particular mathematical form in order to be optimal and logically consistent.

The present work investigates the prognostic power of a set of neuropsychological and Magnetic Resonance Imaging examinations, demographic data, and genetic information about Apolipoprotein-E4The present work investigates the prognostic power of a set of neuropsychological and Magnetic Resonance Imaging examinations, demographic data, and genetic information about Apolipoprotein-E4 (APOE) status, for the prediction of the onset of Alzheimer's Disease in patients defined as mildly cognitively impaired at a baseline examination. The longitudinal data used come from the ADNI database.

 (APOE) status, for the prediction of the onset of Alzheimer's disease in patients defined as mildly cognitively impaired at a baseline examination. The longitudinal data used come from the ADNI database.

The prognostic power of these predictors is quantified in the form of a combined likelihood for the onset of Alzheimer's disease. As a hypothetical example application of personalized clinical decision making, three patient cases are considered where a clinician starts with prognostic uncertainties, possibly coming from other tests, of 50\%/50\%, 25\%/75\%, 75\%/25\%. It is shown how these pre-test probabilities are changed by the predictors. \mynotew{update this}

\mynotew{rewrite following} This quantification also allows us to rank the relative prognostic power of the predictors. It is found that several neuropsychological examinations have the highest prognostic power, much higher than the genetic and imaging-derived predictors included in the present set.

Several additional advantages of this quantification framework are also exemplified and discussed in the present work:
\begin{itemize}
\item missing data are automatically handled, and results having partial data are not discarded; this quantification, therefore, also accounts for patient-dependent availability of \emph{non-independent} predictors;
\item no modelling assumptions (e.g.,\ linearity, gaussianity, functional dependence) are made;
\item the prognostic power obtained is intrinsic to the predictors, that is, it is a bound for \emph{any} prognostic algorithm;
\item variability ranges of the results owing to the finite size of the sample data are automatically quantified.
\item the values obtained, being probabilities, are more easily interpretable than scores of various kinds.
\end{itemize}


%Alzheimer's disease (AD) is by far the most common type of dementia. The disease is characterized by an insidious onset caused by neurodegenerative processes, which lead to progressive loss of cognitive and functional abilities. Alongside the devastating personal consequences AD has on those affected and their caregivers, economical costs related to the disease are massive.
% 
% One of the difficulties for successful treatment of AD is the fact that its pathological hallmarks tend to be established in the brain decades prior to the time a person's cognitive and functional impairments are severe enough to get medical attention. Management of known risk factors for AD (e.g., high blood pressure and diabetes) is therefore emphasized. Recent studies point towards promising life-style interventions reducing AD-pathology and neurodegeneration and delaying symptom-onset. Much effort is therefore put into early identification and treatment of patients in the prodromal phase of the disease. Mild Cognitive Impairment (MCI) has become a diagnostic concept to describe this phase. Individuals falling within this diagnostic category show a cognitive decline greater than expected in normal cognitive aging, but still not with the severity of functional impairment characterizing those with dementia.


% \color{yellow}{\tiny For full guidelines regarding your manuscript please refer to \href{http://www.frontiersin.org/about/AuthorGuidelines}{Author Guidelines}.

% As a primary goal, the abstract should render the general significance and conceptual advance of the work clearly accessible to a broad readership. References should not be cited in the abstract. Leave the Abstract empty if your article does not require one, please see \href{http://www.frontiersin.org/about/AuthorGuidelines#SummaryTable}{Summary Table} for details according to article type.

% } 


\tiny
 \keyFont{\section{Keywords:} Clinical decision making, Utility theory, Probability theory, Artificial Intelligence, Machine Learning, Base-rate fallacy} %All article types: you may provide up to 8 keywords; at least 5 are mandatory.
\end{abstract}

\section{Each patient is unique}

%%%% EXAMPLE-PATIENT DATA %%%%
% Olivia, Ariel, Bianca:
% Apoe4_             0.00000000
% ANARTERR_neuro    18.00000000
% AVDEL30MIN_neuro   5.00000000 Ravlt-del
% AVDELTOT_neuro    10.00000000 Ravlt-rec
% CATANIMSC_neuro   21.00000000
% GDTOTAL_gds        3.00000000
% RAVLT_immediate   36.00000000
% TRAASCOR_neuro    21.00000000
% TRABSCOR_neuro   114.00000000
% AGE               75.38823609
% LRHHC_n_long       0.00425972
% Gender_num_        1.00000000
%
% posterior Olivia, Bianca:  0.301679
% prior Ariel: 0.65
% posterior Ariel: 0.47294
%%%%%%%%%
% Curtis:
% Apoe4_             1.00000000
% ANARTERR_neuro    15.00000000
% AVDEL30MIN_neuro   0.00000000
% AVDELTOT_neuro     3.00000000
% CATANIMSC_neuro   14.00000000
% GDTOTAL_gds        2.00000000
% RAVLT_immediate   20.00000000
% TRAASCOR_neuro    36.00000000
% TRABSCOR_neuro   126.00000000
% AGE               63.82721344
% LRHHC_n_long       NA
% Gender_num_        0.00000000
%
% posterior 0.70256
%
%%%%%%%%
% Utility matrix Olivia, Ariel, Curtis:
% #1  1.0  0.0
% #2  0.9  0.3
% #3  0.8  0.5
% #4  0.0  1.0
%%%%
% Utility matrix Bianca:
% #1  1.0  0.0
% #2  0.8  0.3
% #3  0.7  0.5
% #4  0.0  1.0
%%%%%%%%
% optimal treatments
% Olivia #2
% Ariel #3
% Bianca #1
% Curtis #4

Meet Olivia, Ariel, Bianca, Curtis.\footnote{Fictive characters; any reference to real persons is purely coincidental}
% names from Shakespeare's plays
These four persons don't know each other, but they have something in common: they all suffer from a mild form of cognitive impairment, and are afraid that their impairment will turn into Alzheimer's Disease within a couple of years. In fact, this is why they recently underwent some clinical analyses and cognitive tests. Today they received the results of their analyses. From these results, available clinical statistical data, and other relevant information, their clinician will assess their risk of developing Alzheimer. The clinician and each patient will then decide among a set of possible preventive treatments.

Besides this shared condition and worry, these patients have other things in common -- but also some differences. Let's take Olivia as reference and list the similarities and difference between her and the other three:
\begin{itemize}
\item Olivia and Ariel turn out to have exactly identical clinical results and age. They would also get similar benefits from the available preventive-treatment options. Ariel, however, comes from a different geographical region with a higher rate of conversion, and from a family with a heavy history of Alzheimer's Disease, unlike Olivia. Because of this family background, the clinician judges a priori a 65\% probability that Ariel's cognitive impairment will convert to Alzheimer's Disease

\item Olivia and Bianca also have exactly the same clinical results and age. They come from the same geographical region and have very similar family histories. In fact we shall see that they have the same probability of developing Alzheimer's disease. Bianca, however, suffers from several allergies and additional clinical conditions that render some of the preventive options slightly riskier for her.

\item Olivia and Curtis have different clinical results. In particular, Olivia does not have the risky Apolipoprotein-E4 (APOE4) allele \citep{liuetal2013} whereas Curtis has, and Olivia is more than 10 years older than Curtis. But they otherwise come from the same geographical region, have very similar family histories, and would get similar benefits from the preventive options. Note that one clinical result of Curtis's (hippocampal volume) is missing.
\end{itemize}

We can categorize these differences as \enquote{difference in auxiliary information} (Olivia and Ariel), \enquote{difference in preventive benefits} (Olivia and Bianca), \enquote{difference in clinical predictors} (Olivia and Curtis). Figure~\ref{fig:OABC} summarizes the similarity and differences between Olivia and the other three patients. Table~\ref{tab:patients_data} reports the clinical results and demographic data common to Olivia, Ariel, Bianca, as well as those of Curtis \mynotew{need to explain the variates and refer to \citep{ryeetal2022}}.
\begin{figure}[!h]% with figure
 \centering\includegraphics[width=0.25\linewidth]{OABC.png}\\
\caption{\mynotep{draft, needs better font sizes}}\label{fig:OABC}
\end{figure}%
\begin{table}[!h]
  \centering
  \footnotesize
  \begin{tabular}{ccccccccccccc}
    Patient&
    Age&Sex&HC${}\cdot 10^{-3}$&APOE4&
    ANART&CFT&GDS&RAVLT-im&RAVLT-del&RAVLT-rec&TMTA&TMTB
    \\[1\jot]
    \emph{Olivia, Ariel, Bianca}&
    75.4&F&4.26&N&
    18&21&3&36&5&10&21&114
    \\
    \emph{Curtis}&
    63.8&M&[NA]&Y&
    15&14&2&20&0&3&36&126
  \end{tabular}
  \caption{\mynotew{Clinical results \& demographic data}}\label{tab:patients_data}
\end{table}

\medskip

Considering the similarities and differences among these patients, which treatments are optimal and should prescribed to them?

\medskip

Our main purpose in the present work is to illustrate, using the four fictitious patients above as example, how this clinical decision-making problem can today be solved methodically, exactly, and at low computational cost, when the available prognostic clinical information involves one-dimensional or categorical variates such as those listed in table~\ref{tab:patients_data}. The solution method integrates available clinical statistical data with each new patient's unique combination of clinical results, auxiliary information, and treatment benefits. % It is therefore the staple method for personalized prognosis, diagnosis, treatment.

In our example we shall find that -- despite the many factors in common among our four patients, even despite the identical clinical results for Olivia, Ariel, Bianca, and despite the identical probability of conversion for Olivia and Bianca -- \emph{the optimal treatment option for each patient is different from those for the other three}. This result exemplifies the importance of differences among patients with regard to clinical results, auxiliary information, or preventive benefits.

The method used is none other than decision theory, the combination of probability theory and utility theory \citep[\chaps~13--14]{vonneumannetal1944_r1955,raiffaetal1961_r2000,raiffa1968_r1970,lindley1971_r1988,kreps1988,jaynes1994_r2003}. Medicine has the distinction of having been one of the first fields to adopt it \citep{ledleyetal1959}, with old and new brilliant textbooks \citep{weinsteinetal1980,soxetal1988_r2013,huninketal2001_r2014} that explain and exemplify its application.

Decision theory is also the normative foundation for the construction of an Artificial Intelligence agent capable of rational inference and decision making \citetext{\citealp[\chap~IV]{russelletal1995_r2022}; \citealp[\chaps~1--2]{jaynes1994_r2003}}. The present method can therefore be seen as the application of an \emph{ideal machine-learning algorithm}. \enquote{Ideal} in the sense of being free from approximations, special modelling assumptions, and limitations in its informational output; not in the sense of being impracticable. Another important point of the present work, indeed, is to show that \emph{for some kinds of dataset} such ideal machine-learning algorithm is a reality. It is preferable to popular algorithms such as neural networks, random forests, support-vector machines, which are unsuited to clinical decision-making problems owing to their output limitations. We discuss this matter further in \sect\mynotew{***}.

%%%% Note on "predictand"
%
% The quantity that we want to forecast is in various texts called "dependent variable" or "response variable". I personally don't like either. Both can be misleading. Surely AD-conversion is not "dependent" on cognitive variates, for example. Moreover we'll see that we are actually swapping the role of "independent" and "dependent" variables in this work. Same goes for "response". With a readership of medical scientists it's best to avoid the special connotations of these words, leaving them for variables that are indeed biologically dependent.
%
% This leaves us with "predictand", literally "the thing that has to be predicted", which is exactly what AD-conversion is. I believe this term is used in climate science and meteorology. It's good because it does not misleadingly imply that AD-conversion biologically depends or is a consequence of other variables.
%%%%

\mynotep{Add other advantages of this exact approach:
  \begin{itemize}
  \item it does not make assumptions, besides natural assumption of smoothness of full-population frequency distribution
  \item can be used with partially missing clinical data
  \item can be used with binary or continuous predictands
  \item it tells us the maximum predictive power of the predictors
  \item it quantifies how prediction could change if we had more sample data
  \item it can be applied on-the-fly to each new patient
  \end{itemize}
  and goals, results, and some synopsis}

The inferential and decision-making steps are summarized in table~\ref{tab:inference_decision_steps}.
\begin{table}[!h]  
  \centering
  \begin{framed}
    \caption{Inferential and decision-making steps. Steps in \textbf{boldface} represent patient-dependent, personalized steps that cannot be obtained from the learning dataset}\label{tab:inference_decision_steps}

    \vspace{1em}
    
    \begin{enumerate}\itemsep1em 
    \item\label{item:learn} Infer the full-population frequencies of predictors and predictand, using available datasets.
    \item\label{item:population} \textbf{Assess in respect of which variates the present patient can be considered as belonging to the same population underlying the learning dataset.}
    \item\label{item:prior} \textbf{Assess the prior probability of the predictand for the present patient.} This step allows us (a) to consider additional clinical information outside of the dataset's variates, and available for the present patient only; (b) to correct for mismatches between the dataset's underlying population and the patient's one.
      
    \item\label{item:likelihood}Calculate the \emph{likelihood}
      % -- as opposed to the probability --
      of the predictand for a specific patient, given the patient's predictor values. Combine this likelihood with the prior from step~\ref{item:prior}, to obtain the final probability for the predictand, for the present patient.

    \item\label{item:utilities} \textbf{Assess the clinical courses of action available for the present patient, together their benefits and costs.} This step is fundamentally patient-dependent and is the one open to most variability from patient to patient.

    \item\label{item:expected_utility} Choose the course of action having maximal expected benefit for the present patient, given the benefits assessed in step~\ref{item:utilities} and the final probability assessed in step~\ref{item:likelihood}.  
    \end{enumerate}
  \end{framed}
\end{table}

\section{\protect\mynotep{Practical example}}
\label{sec:example}

\mynotez{It may be optimal to present the steps in reverse order: the last one explains the goal, and makes clear why the preceding steps are necessary.}

\subsection{Learning}
\label{sec:learning_step}

In the learning stage we infer the statistical relationships of a large population of which our future patients can be considered members, at least in some respects. Such relationships will help us in our prognoses. The basic idea is intuitive. If a patient can be considered a member of this population, and if we knew the joint frequencies of all possible combinations of predictor and predictand values in such population -- and knew nothing else -- then we would say that the probability for the patient to have particular values is equal to the population frequency. Pure symmetry considerations lead to this intuitive result \citep[\sects~4.2--4.3]{definetti1930,dawid2013,bernardoetal1994_r2000}.

But it must be emphasized, and it is essential for our method, that it is \emph{not} necessary (and is seldom true) that a future patient be considered as a member of such a population \emph{in all respects}. A patient can be considered a member only \emph{conditionally} on particular variate values. We shall discuss this point with an example in \sect~\ref{sec:population_step}.

If the full statistics of such a population were known, our task would just be to \enquote{enumerate} rather than to \enquote{learn}. Learning comes into play because the full population is not known: we only have a sample from it.

The most we can do is therefore to assign a probability to each possible frequency distribution for the full population. The probability of a specific \enquote{candidate} frequency distribution is intuitively determined by two factors: (a) how well it fits the sample data, (b) how biologically or physically reasonable it is. Figure~\ref{fig:inferring_distribution} show a fictitious sample data and various candidate frequency distributions \mynotep{...}.
\begin{figure}[!t]% with figure
  \centering%
  \includegraphics[width=0.49\linewidth]{exampledistr_sample_all.pdf}
  \hfill
  \includegraphics[width=0.49\linewidth]{exampledistr_okish_all.pdf}
  \\
  \includegraphics[width=0.49\linewidth]{exampledistr_unlikely_all.pdf}
  \hfill
  \includegraphics[width=0.49\linewidth]{exampledistr_strange_all.pdf}
  \caption{\mynotep{Upper-left: Sample data.
      Upper-right: candidate frequency distribution that fits the data and does not look unnatural.
      Lower-left: candidate distribution that might look natural but doesn't fit the sample data.
      Lower-right: candidate distribution that fits the data very well but looks unnatural.}}\label{fig:inferring_distribution}
\end{figure}%

\mynotep{Some more intuition and details about the maths, principles, and characteristics} \citep{dunsonetal2011,rossi2014,rasmussen1999}.
\begin{figure}[!t]% with figure
  \centering%
  \includegraphics[width=0.75\linewidth]{priorexamples_AVDEL30MIN_neuro.pdf}
  \caption{\mynotep{Examples of a-priori probable candidates of frequency distribution for a variate such as \textsf{AVDEL30MIN\_neuro} or \textsf{AVDELTOT\_neuro}}}\label{fig:prior_distribution}
\end{figure}%

The core of our method is the computation of the probabilities of all possible frequency distributions for the full population. From it we obtain the joint probability distribution $\p(X,Y,Z,\dotsc)$ for all variates $X,Y,Z,\dotsc$ available in the dataset.

This is the maximal amount of information that can be extract from our dataset. From it we can indeed quickly calculate any quantity typically outputted by specific or approximate algorithms. For example:
\begin{itemize}
\item \emph{Conditional probability, \enquote{discriminative} algorithms:} if we are interested in the probability of $Z$ given $X,Y$, we calculate $\p(Z \| X,Y) \defd \p(X,Y,Z)/\sum_{Z}\p(X,Y,Z)$.
\item \emph{Conditional probability, \enquote{generative} algorithms:} if we are interested in the probability of $X,Y$ given $Z$, we calculate $\p(X,Y\|Z) \defd \p(X,Y,Z)/\sum_{X,Y}\p(X,Y,Z)$.
\item \emph{Regression or classification:} if we are interested in the average value of $Z$ given $X,Y$, we calculate $\E(Z \| X,Y) \defd \sum_{Z}Z\,\p(Z\|X,Y)$. The \enquote{noise} around this average value is moreover given by $\p(Z-\E\|X,Y)$.
\item \emph{Functional regression:} if $Z$ turns out to be a function $f$ of $X,Y$, then the probability will be a delta distribution: $\p(Z\|X,Y) = \delt[Z-f(X,Y)]$.
  % We must remember that a function can always be represented by a probability distribution, but not vice versa.\footnote{The function $f\colon x \mapsto y=f(x)$ corresponds to the probability density $\p(y\|x) = \delt[y-f(x)]$, where $\delt$ is a delta distribution.}
  The present ideal machine-learning algorithm thus always recovers a functional relationship if there is one, including its noise distribution.
\end{itemize}
We will see that no one of these quantities can be used alone to solve our clinical decision problem for all future patients.

The present method moreover has some further advantages and yields additional useful information:
\begin{itemize}
\item \emph{Discrete or continuous variates:} the variate to be prognosed can be not only binary or discrete, as in the present case, but also continuous.
\item \emph{Partially missing data:} data having missing values for some variates can be fully used, both in the learning dataset and in the prognostic results of new patients.
\item \emph{Maximal predictive power of the variates:} from the probability distribution $\p(X,Y,Z,\dotsc)$ we can calculate the mutual information between any two sets of variates, for example between $Z$ and $\set{X,Y}$. This quantity tells us what is the maximal predictive power -- which can be measured by accuracy or other metrics -- from one set to the other that can be achieved by any inference algorithm \citep{mackay1995_r2005,goodetal1968,coveretal1991_r2006}. Functional dependence is included as a special case; for example, a binary variate $Z$ is a function of $\set{X,Y}$ if and only if their mutual information \footnote{The \enquote{shannon} ($\mathrm{Sh}$) is a measurement unit of information, as specified by the \cite{iso2008c}.} equals $1\,\mathrm{Sh}$.
\item \emph{Variability owing to limited sample size:} we can calculate how much any of the quantities listed so far -- from average values to mutual information -- could change if more sample data were added to our dataset.
\end{itemize}

\medskip

For readers with an interest in machine learning and artificial intelligence, we briefly discuss the drawbacks of some popular inference algorithms with respect to the present ideal algorithm.

\emph{Neural networks and gaussian processes} are based on the assumption that there is a functional relationship from predictors to predictand,
possibly contaminated by a little noise, typically assumed gaussian. This is a very strong assumption, quite unrealistic for many kinds of variables considered in medicine. It can only be justified in the presence of informationally very rich predictors such as images. In our case the mutual information between predictors and the conversion variate is $0.14\,\mathrm{Sh}$, to be compared with $1\,\mathrm{Sh}$, if conversion were a function of the predictors, and with $0\,\mathrm{Sh}$, if conversion were completely unpredictable. See \sect~\mynotep{***} for further details. An additional deficiency of neural networks is that they do not yield any probabilities, even if there are many efforts to render such an output possible \citep{pearceetal2020,osbandetal2021,backetal2019}. Such an advance, however, would still not solve the final deficiency of neural networks and gaussian processes: they try to infer the predictand from the predictors but cannot be used for the reverse inference.

\emph{Random forests} also assume a functional relationship from predictors to predictand. This assumptions is mitigated when the predictand is a discrete variate; in this case a random forest can output an agreement score from its constituent decision trees. This score can give an idea of the underlying uncertainty, but it is not a probability,\footnote{It is sometimes called a \enquote{non-calibrated probability}, which is akin to a \enquote{non-round circle}.} and therefore cannot be used in the decision-making stage, see \sect~\ref{sec:expected_utility_step}. It is possible to transform this score into a proper probability \citep{dyrlandetal2022b}, but this possibility does not solve the final deficiency of random forests: like neural networks, they try to infer the predictand from the predictors but cannot be used for the reverse inference.

\emph{Parametric models and machine-learning algorithms} such as logistic or linear regression, support-vector machines, or generalized linear models make even stronger assumptions than neural networks and random forests. They assume specific functional shapes or frequency distributions. Their use may be justified when we are extremely sure -- for instance thanks to underlying physical or biological knowledge -- of the validity of their assumptions; or when the computational resources are extremely scarce. But it is otherwise unnecessary to be hampered by their restrictive and often unrealistic assumptions.

\medskip

An important common deficiency of most inference algorithms mentioned above is that their inference only goes from predictors to predictands. In the next two sections we shall see that this precludes -- or makes much riskier -- the prognostic use of the learning dataset for patients belonging to different populations, such as Ariel.


%%%% MI given all minus ...
%      AVDEL30MIN_neuro RAVLT_immediate TRABSCOR_neuro AVDELTOT_neuro LRHHC_n_long
% mean            0.125           0.131          0.135          0.137        0.138
% sd              0.004           0.004          0.004          0.004        0.004
%      TRAASCOR_neuro CATANIMSC_neuro   AGE GDTOTAL_gds ANARTERR_neuro Gender_num_
% mean          0.139           0.139 0.139       0.140          0.140       0.140
% sd            0.004           0.004 0.004       0.004          0.004       0.004
%      Apoe4_   all
% mean  0.140 0.140
% sd    0.004 0.004


%%%% MI differences: 
%        all    Apoe4_ Gender_num_ ANARTERR_neuro GDTOTAL_gds      AGE CATANIMSC_neuro
% mean 0.000 0.0000121   0.0000939       0.000264    0.000295 0.000512        0.000561
% sd   0.009 0.0090000   0.0090000       0.009000    0.009000 0.009000        0.009000
%      TRAASCOR_neuro LRHHC_n_long AVDELTOT_neuro TRABSCOR_neuro RAVLT_immediate
% mean        0.00113      0.00187        0.00234        0.00438         0.00905
% sd          0.00900      0.00900        0.00900        0.00900         0.00800
%      AVDEL30MIN_neuro
% mean           0.0144
% sd             0.0080



\subsection{Population assessment and prior probability}
\label{sec:population_step}

\setlength{\intextsep}{0ex}% with wrapfigure
\setlength{\columnsep}{1ex}% with wrapfigure
\begin{wrapfigure}{r}{0.25\linewidth}% with wrapfigure
%\vspace{-1ex}%
\includegraphics[width=\linewidth]{baseratefallacy3.png}%
% \includegraphics[width=\linewidth]{baseratetrials.png}\\[\jot]
% \includegraphics[width=\linewidth]{baseratepopulation.png}
\end{wrapfigure}%
Most medicine students learn about the \emph{base-rate fallacy} \citep{barhillel1980,jennyetal2018,sprengeretal2021,matthews1996}. Consider a large set of clinical trials, illustrated in the upper table on the side, where each dot represents, say, 10\,000 patients. In this sample dataset it is found that, among patients having a particular value \enquote{+} of some predictors (left column), 71.4\% (or 5/7, upper square) of them eventually developed a disease. The fallacy lies in judging that a new real patient from the full population, who has that particular predictor value, also has a 71.4\% probability of developing that disease. In fact, \emph{this probability will in general be different}. In our example it is 33.3\% (5/15), as can be seen in the lower table illustrating the full population. This difference would be noticed as soon as the inappropriate probability were used to make prognoses in the full population.

There is a discrepancy in the frequencies of predictand given predictors for the sample dataset and for the full population, because the proportion of positive vs negative disease cases in the latter has some value, 16.7\%/83.3\% in our example, whereas the samples for the trials (dashed line in the lower table) were chosen so as to have a 50\%/50\% proportion. This sampling procedure is called \enquote{class balancing} in machine learning \citep{provost2000,drummondetal2005,weissetal2003}. More generally this discrepancy can appear whenever a population and a sample dataset from it do not have the same frequency distribution for the predictand. In this case we cannot rely on the probabilities of \enquote{predictand given predictors} obtained from the sample dataset.

A little counting in the side figure reveals, however, that other frequencies may be relied upon. Consider the full population. Among all patients who developed the disease, 83.3\% (or 5/6, upper row) of them had the particular predictor value, while among those who did not develop the disease, 33.3\% (or 1/3, lower row) had the particular predictor value. \emph{And these frequencies are the same in the sample dataset}. These frequencies from the clinical trials can therefore be used to make a prognosis using Bayes's theorem:
\begin{equation}
  \label{eq:base-rate_correction}
  % \p(\textsf{\small predictand} \| \textsf{\small predictors}) \propto
  % \p(\textsf{\small predictors} \| \textsf{\small predictand},
  % \textsf{\small dataset})
  % \cdot   \p(\textsf{\small predictand} \| \textsf{\small population})
  \p(\textsf{\small predictand} \| \textsf{\small predictors}) =
  \frac{
    \p(\textsf{\small predictors} \| \textsf{\small predictand},
  \textsf{\small dataset})
  \cdot   \p(\textsf{\small predictand} \| \textsf{\small population})
}{\sum\limits_{\text{predictand}}
    \p(\textsf{\small predictors} \| \textsf{\small predictand},
  \textsf{\small dataset})
  \cdot   \p(\textsf{\small predictand} \| \textsf{\small population})
}
\end{equation}
In our example we find
\begin{equation}
  \label{eq:base-rate_correction_example}
 \begin{split}
  % \p(\textsf{\small predictand} \| \textsf{\small predictors}) \propto
  % \p(\textsf{\small predictors} \| \textsf{\small predictand},
  % \textsf{\small dataset})
  % \cdot   \p(\textsf{\small predictand} \| \textsf{\small population})
   \p(\textsf{\small disease+} \| \textsf{\small predictor+})
   &=
  \frac{
    \p(\textsf{\small predictor+} \| \textsf{\small disease+},
  \textsf{\small trials})
  \cdot   \p(\textsf{\small disease+} \| \textsf{\small population})
}{
  \biggl[\begin{aligned}
    &\p(\textsf{\small predictor+} \| \textsf{\small disease+},
  \textsf{\small trials}) 
  \cdot   \p(\textsf{\small disease+} \| \textsf{\small population})
  +{}\\[-0.5\jot]&\hspace{5em}
    \p(\textsf{\small predictor+} \| \textsf{\small disease\textminus},
  \textsf{\small trials})
  \cdot   \p(\textsf{\small disease\textminus} \| \textsf{\small population})
\end{aligned}\biggr]
} \\[2\jot]
&\approx
  \frac{ 0.833 \cdot 0.167}{0.833 \cdot 0.167 + 0.333 \cdot 0.833}
  = 0.33
  \end{split}
\end{equation}
which is indeed the correct full-population frequency.

If the samples of the clinical trials had been chosen with the same frequencies as the full population (no \enquote{class balancing}), then the probability $\p(\textsf{\small predictand} \| \textsf{\small predictors}, \textsf{\small dataset})$ from the dataset would be the appropriate one to use. But the probabilities $\p(\textsf{\small predictors} \| \textsf{\small predictand}, \textsf{\small dataset})$ together with Bayes's theorem as in \eqn~\eqref{eq:base-rate_correction} would also lead to exactly the same probability. We thus see that \emph{using the probabilities}
\[\p(\textsf{\small predictors} \| \textsf{\small predictand}, \textsf{\small dataset})\]
\emph{from the dataset is preferable to using} $\p(\textsf{\small predictand} \| \textsf{\small predictors}, \textsf{\small dataset})$. The former yield the same results as the latter when use of the latter is appropriate, and allow us to apply corrections when use of latter is inappropriate.

The use of dataset probabilities different from $\p(\textsf{\small predictand} \| \textsf{\small predictors}, \textsf{\small dataset})$ can be necessary even when the dataset has statistics identical with the population it is sampled from. Typical cases are the prognosis of a patient that comes from a peculiar subpopulation or even from a different population. The first case happens for instance when the clinician has additional information not included among the predictor variates, such as the result of an additional clinical test, or family history. The second case happens for instance when the patient comes from a different geographical region. There is of course no sharp distinction between these two cases.

What is important is that in either case it can still be possible to use statistical information from the sample dataset to make prognoses. It is sufficient that some \emph{conditional} may be applicable to the specific patient. For a patient coming from a different region, for example, it may be judged that the conditional probabilities $\p(\textsf{\small predictand} \| \textsf{\small predictors}, \textsf{\small dataset})$ still apply. Stated otherwise the patient may still be considered of a member of the subpopulation having those specific predictor values.

This topic is complex and its study is not the goal of the present work. We refer the readers to the brilliant paper by Lindley \amp\ Novick \citeyearpar{lindleyetal1981} for further discussion, and the works by \cite{malinasetal2004_r2016} and \cite{sprengeretal2021} about Simpson's paradox, to which this topic is related. \mynotez{Maybe add refs to Pearl (and Russell) about the notion of causality -- which is somewhat circular, however.}

Our main point is that subpopulation or population variability is an important factor that differentiate patients, and it should be taken into account in a personalized approach. The method presented here does this naturally, allowing a great flexibility in picking which statistical features of the sample dataset should be used for each new patients. As discussed in \sect~\ref{sec:learning_step}, the ideal machine-learning algorithm used here allow us to quickly calculate conditional probabilities $\p(Y\|X, \textsf{\small dataset})$ for any desired variate subsets $Y$ and $X$, as required by a specific patient's population.

\mynotew{Illustration of this: We split our learning dataset in two subsets}

% The situation discussed above generalizes and becomes more complicated as we consider multiple predictor and predictand variates. In general, of all probabilities $\p(\dotso\|\dotso)$ obtained from the dataset we should use those having in the conditional %(
% \enquote{$\|\dotso)$} any variate suspected to have dataset statistics different from those of the population of interest. The final desired probabilities are then obtained through Bayes's theorem, supplying additional corrected statistics.

% Let's clarify this with an example. In our dataset we observe a value $\textsf{GDS} = 1$ with a frequency of 32\% across all patients. Among the patients who later converted to \ad, the frequency is 36\%; whereas among those who remained with a stable \mci, the frequency is 28\%. Given a new patient, if we knew that 

In our present example, Ariel

\subsection{Likelihood and posterior probability}
\label{sec:posterior_step}

 \citep{lindleyetal1981,sprengeretal2021,barhillel1980}


\subsection{Benefit assessments}
\label{sec:utilities_step}

 \citep{soxetal1988_r2013,huninketal2001_r2014}

\subsection{Maximization of expected benefit}
\label{sec:expected_utility_step}

 \citep{lindley1982}


\section{Discussion}
\label{sec:discussion}

\mynotez{Difficulty in assessing and quantifying additional info: Just ignoring it is not a solution and is unethical}

\mynotez{Subtly hidden disastrous consequences of not following normative decision theory: An algorithm can lead to saving 85\,000 patients out of 100\,000 and be deemed a success. But if the ideal algorithm had been used, 95\,000 patients would actually have been saved. What shall we say to the families of the 10\,000 patients who could have been saved but weren't?}






%%\setlength{\intextsep}{0ex}% with wrapfigure
%%\setlength{\columnsep}{0ex}% with wrapfigure
%\begin{figure}[p!]% with figure
%\begin{wrapfigure}{r}{0.4\linewidth} % with wrapfigure
%  \centering\includegraphics[trim={12ex 0 18ex 0},clip,width=\linewidth]{maxent_saddle.png}\\
%\caption{caption}\label{fig:comparison_a5}
%\end{figure}% exp_family_maxent.nb




\newpage
\hrule
\hrule

--- Luca, old pieces of text ---

Personalized diagnosis, prognosis, treatment, and prevention strategies must make allowance for several fundamental differences among patients:
\begin{itemize}
\item the \emph{kinds} of clinical data and evidence available for diagnosis or prognosis can be different;
\item the \emph{values} of the same kind of clinical data and evidence can be different;
\item the kinds of treatment or prevention options can be different;
\item the advantages and disadvantages, benefits and costs of the same kinds of treatment or prevention can be different;
\item finally, the evaluation of the clinician -- which often relies on too subtle clues (family history, regional history, case experience) to be considered as measurable data -- can be different.
\end{itemize}

Is there really a methodological framework that can take all these differences into account? Yes, there is, and Medicine has the distinction of having been one of the first fields to adopt it \citep{ledleyetal1959}: Statistical Decision Theory. Its application in Medicine is explained and exemplified in several, brilliant, old and new textbooks \citep{weinsteinetal1980,soxetal1988_r2013,huninketal2001_r2014}. This theory has mathematical and logical foundations and its principles constitute indeed the foundations for the definition and realization of Artificial Intelligence \citep{russelletal1995_r2022}
\mynotew{}



The basics of clinical decision making \mynotew{..basics: each piece of evidence contributes with a likelihood or odds; they combine together and together with the clinician's pre-data evaluation. Then they are combined with the different benefits/costs of treatments or prevention strategies to find the optimal one. Decision trees can be necessary (but don't change this framework). Costs \& benefits are evaluated by clinician \& patient together.}


\begin{multline}\label{eq:bayes_theorem_intro}
    \overbrace{\vphantom{\bigg\{}\p(\text{health condition}\|
      \text{results of all tests},\ 
      \text{prior info})}^{\zerob{\scriptsize\emph{post-test probability}}}
    \mathrel{\ \propto\ }{}
    \\
    \shoveright{\overbrace{\vphantom{\bigg\{}\p(\text{health condition} \|
      \text{prior info})}^{\zerob{\scriptsize\emph{pre-test probability by clinician}}}
  \mathbin{\ \times\ }{}}    
    \\[\jot]
\text{\scriptsize\emph{likelihoods of tests}}\left\{\;  \begin{aligned}
 \p(\text{result of 1st test} \| \text{health condition} ,\ 
  \text{prior info})
  &\mathbin{\ \times\ }{}\\
  \p(\text{result of 2nd test} \| \text{health condition} ,\ 
  \text{prior info})
  &\mathbin{\ \times\ }{}
\\
{}\dotsb &
  \end{aligned}\right.
\end{multline}


% {\color{yellow}\tiny For Original Research Articles \citep{conference}, Clinical Trial Articles \citep{article}, and Technology Reports \citep{patent}, the introduction should be succinct, with no subheadings \citep{book}. For Case Reports the Introduction should include symptoms at presentation \citep{chapter}, physical exams and lab results \citep{dataset}.

% }



\section{Article types}

For requirements for a specific article type please refer to the Article Types on any Frontiers journal page. Please also refer to  \href{http://home.frontiersin.org/about/author-guidelines#Sections}{Author Guidelines} for further information on how to organize your manuscript in the required sections or their equivalents for your field

% For Original Research articles, please note that the Material and Methods section can be placed in any of the following ways: before Results, before Discussion or after Discussion.

\section{Manuscript Formatting}

\subsection{Heading Levels}

%There are 5 heading levels

\subsection{Level 2}
\subsubsection{Level 3}
\paragraph{Level 4}
\subparagraph{Level 5}

\subsection{Equations}
Equations should be inserted in editable format from the equation editor.

\begin{equation}
\sum x+ y =Z\label{eq:01}
\end{equation}

\subsection{Figures}
Frontiers requires figures to be submitted individually, in the same order as they are referred to in the manuscript. Figures will then be automatically embedded at the bottom of the submitted manuscript. Kindly ensure that each table and figure is mentioned in the text and in numerical order. Figures must be of sufficient resolution for publication \href{https://www.frontiersin.org/about/author-guidelines#ImageSizeRequirements}{see here for examples and minimum requirements}. Figures which are not according to the guidelines will cause substantial delay during the production process. Please see \href{https://www.frontiersin.org/about/author-guidelines#FigureRequirementsStyleGuidelines}{here} for full figure guidelines. Cite figures with subfigures as figure \ref{fig:Subfigure 1} and \ref{fig:Subfigure 2}.


\subsubsection{Permission to Reuse and Copyright}
Figures, tables, and images will be published under a Creative Commons CC-BY licence and permission must be obtained for use of copyrighted material from other sources (including re-published/adapted/modified/partial figures and images from the internet). It is the responsibility of the authors to acquire the licenses, to follow any citation instructions requested by third-party rights holders, and cover any supplementary charges.
%%Figures, tables, and images will be published under a Creative Commons CC-BY licence and permission must be obtained for use of copyrighted material from other sources (including re-published/adapted/modified/partial figures and images from the internet). It is the responsibility of the authors to acquire the licenses, to follow any citation instructions requested by third-party rights holders, and cover any supplementary charges.

\subsection{Tables}
Tables should be inserted at the end of the manuscript. Please build your table directly in LaTeX.Tables provided as jpeg/tiff files will not be accepted. Please note that very large tables (covering several pages) cannot be included in the final PDF for reasons of space. These tables will be published as \href{http://home.frontiersin.org/about/author-guidelines#SupplementaryMaterial}{Supplementary Material} on the online article page at the time of acceptance. The author will be notified during the typesetting of the final article if this is the case. 

\section{Nomenclature}

\subsection{Resource Identification Initiative}
To take part in the Resource Identification Initiative, please use the corresponding catalog number and RRID in your current manuscript. For more information about the project and for steps on how to search for an RRID, please click \href{http://www.frontiersin.org/files/pdf/letter_to_author.pdf}{here}.

\subsection{Life Science Identifiers}
Life Science Identifiers (LSIDs) for ZOOBANK registered names or nomenclatural acts should be listed in the manuscript before the keywords. For more information on LSIDs please see \href{https://www.frontiersin.org/about/author-guidelines#Nomenclature}{Inclusion of Zoological Nomenclature} section of the guidelines.


\section{Additional Requirements}

For additional requirements for specific article types and further information please refer to \href{http://www.frontiersin.org/about/AuthorGuidelines#AdditionalRequirements}{Author Guidelines}.

\section*{Conflict of Interest Statement}
%All financial, commercial or other relationships that might be perceived by the academic community as representing a potential conflict of interest must be disclosed. If no such relationship exists, authors will be asked to confirm the following statement: 

The authors declare that the research was conducted in the absence of any commercial or financial relationships that could be construed as a potential conflict of interest.

\section*{Author Contributions}

The authors were too immersed in the development of the present work to keep a detailed record of who did what.

% The Author Contributions section is mandatory for all articles, including articles by sole authors. If an appropriate statement is not provided on submission, a standard one will be inserted during the production process. The Author Contributions statement must describe the contributions of individual authors referred to by their initials and, in doing so, all authors agree to be accountable for the content of the work. Please see  \href{https://www.frontiersin.org/about/policies-and-publication-ethics#AuthorshipAuthorResponsibilities}{here} for full authorship criteria.

\section*{Funding}
Details of all funding sources should be provided, including grant numbers if applicable. Please ensure to add all necessary funding information, as after publication this is no longer possible.

\section*{Acknowledgments}
This is a short text to acknowledge the contributions of specific colleagues, institutions, or agencies that aided the efforts of the authors.

\section*{Supplemental Data}
 \href{http://home.frontiersin.org/about/author-guidelines#SupplementaryMaterial}{Supplementary Material} should be uploaded separately on submission, if there are Supplementary Figures, please include the caption in the same file as the figure. LaTeX Supplementary Material templates can be found in the Frontiers LaTeX folder.

\section*{Data Availability Statement}
The datasets [GENERATED/ANALYZED] for this study can be found in the [NAME OF REPOSITORY] [LINK].
% Please see the availability of data guidelines for more information, at https://www.frontiersin.org/about/author-guidelines#AvailabilityofData

\bibliographystyle{Frontiers-Harvard} %  Many Frontiers journals use the Harvard referencing system (Author-date), to find the style and resources for the journal you are submitting to: https://zendesk.frontiersin.org/hc/en-us/articles/360017860337-Frontiers-Reference-Styles-by-Journal. For Humanities and Social Sciences articles please include page numbers in the in-text citations 
%\bibliographystyle{Frontiers-Vancouver} % Many Frontiers journals use the numbered referencing system, to find the style and resources for the journal you are submitting to: https://zendesk.frontiersin.org/hc/en-us/articles/360017860337-Frontiers-Reference-Styles-by-Journal
\bibliography{portamanabib}

%%% Make sure to upload the bib file along with the tex file and PDF
%%% Please see the test.bib file for some examples of references

%% \section*{Figure captions}

%%% Please be aware that for original research articles we only permit a combined number of 15 figures and tables, one figure with multiple subfigures will count as only one figure.
%%% Use this if adding the figures directly in the mansucript, if so, please remember to also upload the files when submitting your article
%%% There is no need for adding the file termination, as long as you indicate where the file is saved. In the examples below the files (logo1.eps and logos.eps) are in the Frontiers LaTeX folder
%%% If using *.tif files convert them to .jpg or .png
%%%  NB logo1.eps is required in the path in order to correctly compile front page header %%%

% \begin{figure}[h!]
% \begin{center}
% \includegraphics[width=10cm]{logo1}% This is a *.eps file
% \end{center}
% \caption{ Enter the caption for your figure here.  Repeat as  necessary for each of your figures}\label{fig:1}
% \end{figure}

% \setcounter{figure}{2}
% \setcounter{subfigure}{0}
% \begin{subfigure}
% \setcounter{figure}{2}
% \setcounter{subfigure}{0}
%     \centering
%     \begin{minipage}[b]{0.5\textwidth}
%         \includegraphics[width=\linewidth]{logo1.eps}
%         \caption{This is Subfigure 1.}
%         \label{fig:Subfigure 1}
%     \end{minipage}  
   
% \setcounter{figure}{2}
% \setcounter{subfigure}{1}
%     \begin{minipage}[b]{0.5\textwidth}
%         \includegraphics[width=\linewidth]{logo2.eps}
%         \caption{This is Subfigure 2.}
%         \label{fig:Subfigure 2}
%     \end{minipage}

% \setcounter{figure}{2}
% \setcounter{subfigure}{-1}
%     \caption{Enter the caption for your subfigure here. \textbf{(A)} This is the caption for Subfigure 1. \textbf{(B)} This is the caption for Subfigure 2.}
%     \label{fig: subfigures}
% \end{subfigure}

%%% If you don't add the figures in the LaTeX files, please upload them when submitting the article.
%%% Frontiers will add the figures at the end of the provisional pdf automatically
%%% The use of LaTeX coding to draw Diagrams/Figures/Structures should be avoided. They should be external callouts including graphics.

\end{document}
