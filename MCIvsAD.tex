%%%%%%%%%%%%%%%%%%%%%%%%%%%%%%%%%%%%%%%%%%%%%%%%%%%%%%%%%%%%%%%%%%%%%%%%%%%%%%%%%%%%%%%%%%%%%%%%%%%%%%%%%%%%%%%%%%%%%%%%%%%%%%%%%%%%%%%%%%%%%%%%%%%%%%%%%%%
% This is just an example/guide for you to refer to when submitting manuscripts to Frontiers, it is not mandatory to use Frontiers .cls files nor frontiers.tex  %
% This will only generate the Manuscript, the final article will be typeset by Frontiers after acceptance.   
%                                              %
%                                                                                                                                                         %
% When submitting your files, remember to upload this *tex file, the pdf generated with it, the *bib file (if bibliography is not within the *tex) and all the figures.
%%%%%%%%%%%%%%%%%%%%%%%%%%%%%%%%%%%%%%%%%%%%%%%%%%%%%%%%%%%%%%%%%%%%%%%%%%%%%%%%%%%%%%%%%%%%%%%%%%%%%%%%%%%%%%%%%%%%%%%%%%%%%%%%%%%%%%%%%%%%%%%%%%%%%%%%%%%

%%% Version 3.4 Generated 2022/06/14 %%%
%%% You will need to have the following packages installed: datetime, fmtcount, etoolbox, fcprefix, which are normally inlcuded in WinEdt. %%%
%%% In http://www.ctan.org/ you can find the packages and how to install them, if necessary. %%%
%%%  NB logo1.jpg is required in the path in order to correctly compile front page header %%%

\documentclass[utf8]{FrontiersinHarvard} % for articles in journals using the Harvard Referencing Style (Author-Date), for Frontiers Reference Styles by Journal: https://zendesk.frontiersin.org/hc/en-us/articles/360017860337-Frontiers-Reference-Styles-by-Journal
%\documentclass[utf8]{FrontiersinVancouver} % for articles in journals using the Vancouver Reference Style (Numbered), for Frontiers Reference Styles by Journal: https://zendesk.frontiersin.org/hc/en-us/articles/360017860337-Frontiers-Reference-Styles-by-Journal
%\documentclass[utf8]{frontiersinFPHY_FAMS} % Vancouver Reference Style (Numbered) for articles in the journals "Frontiers in Physics" and "Frontiers in Applied Mathematics and Statistics" 

%\setcitestyle{square} % for articles in the journals "Frontiers in Physics" and "Frontiers in Applied Mathematics and Statistics" 
\usepackage{url,hyperref,lineno,microtype,subcaption}
\usepackage[onehalfspacing]{setspace}
\usepackage[usenames]{xcolor}
% Tol (2012) colour-blind-, print-, screen-friendly colours, alternative scheme; Munsell terminology
\definecolor{bluepurple}{RGB}{68,119,170}
\definecolor{blue}{RGB}{102,204,238}
\definecolor{green}{RGB}{34,136,51}
\definecolor{yellow}{RGB}{204,187,68}
\definecolor{red}{RGB}{238,102,119}
\definecolor{redpurple}{RGB}{170,51,119}
\definecolor{grey}{RGB}{187,187,187}
\definecolor{lgrey}{RGB}{221,221,221}

\definecolor{notecolour}{RGB}{68,170,153}
%\newcommand*{\puzzle}{\maltese}
\newcommand*{\puzzle}{{\fontencoding{U}\fontfamily{fontawesometwo}\selectfont\symbol{225}}}
\newcommand*{\wrench}{{\fontencoding{U}\fontfamily{fontawesomethree}\selectfont\symbol{114}}}
\newcommand*{\pencil}{{\fontencoding{U}\fontfamily{fontawesometwo}\selectfont\symbol{210}}}
\newcommand{\mynotew}[1]{{\color{notecolour}\wrench\ #1}}
\newcommand{\mynotep}[1]{{\color{notecolour}\pencil\ #1}}
\newcommand{\mynotez}[1]{{\color{notecolour}\puzzle\ #1}}

%\usepackage[shortlabels,inline]{enumitem}
%\SetEnumitemKey{para}{itemindent=\parindent,leftmargin=0pt,listparindent=\parindent,parsep=0pt,itemsep=\topsep}
%\setlist{itemsep=0pt,topsep=\parsep}
%\setlist[enumerate,2]{label=(\roman*)}
%\setlist[enumerate]{label=(\alph*),leftmargin=1.5\parindent}
%\setlist[itemize]{leftmargin=1.5\parindent}
%\setlist[description]{leftmargin=1.5\parindent}

\linenumbers

\usepackage{mathtools}
%% Macros
\DeclarePairedDelimiter\abs{\lvert}{\rvert}
\DeclarePairedDelimiter\set{\{}{\}} %}
\newcommand*{\p}{\mathrm{p}}%probability
\renewcommand*{\P}{\mathrm{P}}%probability
%\newcommand*{\E}{\mathrm{E}}
%% The "\:" space is chosen to correctly separate inner binary and external relationss
\renewcommand*{\|}[1][]{\nonscript\:#1\vert\nonscript\:\mathopen{}}
\newcommand*{\defd}{\coloneqq}
\newcommand*{\defs}{\eqqcolon}
\newcommand*{\Land}{\bigwedge}
\newcommand*{\zerob}[1]{\makebox[0pt][c]{#1}}



% Leave a blank line between paragraphs instead of using \\


\def\keyFont{\fontsize{8}{11}\helveticabold }
\def\firstAuthorLast{Sample {et~al.}} %use et al only if is more than 1 author
\def\Authors{P.G.L. Porta Mana\,$^{1,2,*}$, Ingrid Rye\,$^{3}$, Alexandra Vik\,$^{1,2}$, Marek Kocinski\,$^{2,4}$, Arvid Lundervold\,$^{2,4}$, Astri J. Lundervold\,$^{3}$, and Alexander S. Lundervold\,$^{1,2}$}
% Affiliations should be keyed to the author's name with superscript numbers and be listed as follows: Laboratory, Institute, Department, Organization, City, State abbreviation (USA, Canada, Australia), and Country (without detailed address information such as city zip codes or street names).
% If one of the authors has a change of address, list the new address below the correspondence details using a superscript symbol and use the same symbol to indicate the author in the author list.
\def\Address{$^{1}$Department of Computer Science, Electrical Engineering and Mathematical Sciences, Western Norway University of Applied Sciences, Bergen, Norway \\
$^{2}$Mohn Medical Imaging and Visualization Centre (MMIV), Department of Radiology, Haukeland University Hospital, Bergen, Norway\\
$^{3}$Department of Biological and Medical Psychology, University of Bergen, Norway\\
$^{4}$Department of Biomedicine, University of Bergen, Norway}
% The Corresponding Author should be marked with an asterisk
% Provide the exact contact address (this time including street name and city zip code) and email of the corresponding author
\def\corrAuthor{Corresponding Author}

\def\corrEmail{pgl@portamana.org}




\begin{document}
\onecolumn
\firstpage{1}

%\title[Conversion from MCI to AD]{Conversion from MCI to Alzheimer's disease: model-free predictions with quantified uncertainty} 
%\title[Conversion from MCI to AD]{Model-free predictions with quantified uncertainty in personalized medicine: A case study on the conversion from MCI to AD} 
\title[Conversion from Mild Cognitive Impairment to Alzheimer's Disease]{Model-free prognosis and decision in personalized medicine: A case study on the conversion from Mild Cognitive Impairment to Alzheimer's Disease} 

\author[\firstAuthorLast ]{\Authors} %This field will be automatically populated
\address{} %This field will be automatically populated
\correspondance{} %This field will be automatically populated

\extraAuth{}% If there are more than 1 corresponding author, comment this line and uncomment the next one.
%\extraAuth{corresponding Author2 \\ Laboratory X2, Institute X2, Department X2, Organization X2, Street X2, City X2 , State XX2 (only USA, Canada and Australia), Zip Code2, X2 Country X2, email2@uni2.edu}


\maketitle


\begin{abstract}

%%% Leave the Abstract empty if your article does not require one, please see the Summary Table for full details.
\section{}
[Luca \& Astri, draft]

Patients with Mild Cognitive Impairment (MCI) have an increased risk of a trajectory toward Alzheimer's disease (AD). Early identification of patients with a high risk of AD is
%of underlying neurodegenerative processes 
essential to provide  treatment before the disease is well-established in the brain. 
It is, therefore, of %extreme interest 
great importance to 
study how well different kinds of predictors 
%-- from neuropsychological examinations to advanced brain-imaging techniques -- 
allow us to estimate %prognose 
a trajectory from MCI towards AD in an individual patient.

But more is needed for a personalized approach to prognosis, prevention, and treatment, than just the obvious requirement that prognoses be as best as they can be for each patient. Several situational elements that can be different from patient to patient must be accounted for:
\begin{itemize}
\item the \emph{kinds} of clinical data and evidence available for prognosis;
\item the \emph{outcomes} of the same kind of clinical data and evidence;
\item the kinds of treatment or prevention strategies available, owing to different attitudes toward life, different family networks and possibilities of familial support, different additional medical factors such as physical disabilities, and different economic means;
\item the advantages and disadvantages, benefits and costs of the same kinds of treatment or prevention strategies; the patient has a major role in the quantification of such benefits and costs;
\item finally, the initial evaluation by the clinician -- which often relies on too subtle clues (family history, regional history, previous case experience) to be considered as measurable data.
\end{itemize}
Statistical decision theory is the normative quantification framework that takes into account these fundamental differences. Medicine has the distinction of having been one of the first fields to adopt this framework, exemplified in brilliant old and new textbooks on clinical decision-making. 

Clinical decision-making makes allowance for these differences among patients through two requirements. First, the quantification of prognostic evidence on one side, and of benefits and costs of treatments and prevention strategies on the other, must be clearly separated and handled in a modular way. Two patients can have the same prognostic evidence and yet very different prevention options. Second, the quantification of independent prognostic evidence ought to be in the form of \emph{likelihoods about the health condition} (or equivalently of likelihood ratios, in a binary case), that is, of the probabilities of the observed test outcomes given the hypothesized health conditions. Likelihoods from independent clinical tests and predictors can then be combined with a simple multiplication; for one patient, we could have three kinds of predictor available; for another, we could have five. The clinician's pre-test assessment is included in the form of a probability. These patient-dependent probabilities are combined with the patient-dependent costs and benefits of treatment or prevention to arrive at the best course of action for that patient. The main result underlying statistical decision theory is that decision-making \emph{must} take this particular mathematical form in order to be optimal and logically consistent.

The present work investigates the prognostic power of a set of neuropsychological and Magnetic Resonance Imaging examinations, demographic data, and genetic information about Apolipoprotein-E4 (APOE) status, for the prediction of the onset of Alzheimer's disease in patients defined as mildly cognitively impaired at a baseline examination. The longitudinal data used come from the ADNI database.

The prognostic power of these predictors is quantified in the form of a combined likelihood for the onset of Alzheimer's disease. As a hypothetical example application of personalized clinical decision making, three patient cases are considered where a clinician starts with prognostic uncertainties, possibly coming from other tests, of 50\%/50\%, 25\%/75\%, 75\%/25\%. It is shown how these pre-test probabilities are changed by the predictors.

This quantification also allows us to rank the relative prognostic power of the predictors. It is found that several neuropsychological examinations have the highest prognostic power, much higher than the genetic and imaging-derived predictors included in the present set.

Several additional advantages of this quantification framework are also exemplified and discussed in the present work:
\begin{itemize}
\item missing data are automatically handled, and results having partial data are not discarded; this quantification, therefore, also accounts for patient-dependent availability of \emph{non-independent} predictors;
\item no modelling assumptions (e.g.,\ linearity, gaussianity, functional dependence) are made;
\item the prognostic power obtained is intrinsic to the predictors, that is, it is a bound for \emph{any} prognostic algorithm;
\item variability ranges of the results owing to the finite size of the sample data are automatically quantified.
\item the values obtained, being probabilities, are more easily interpretable than scores of various kinds.
\end{itemize}


%Alzheimer's disease (AD) is by far the most common type of dementia. The disease is characterized by an insidious onset caused by neurodegenerative processes, which lead to progressive loss of cognitive and functional abilities. Alongside the devastating personal consequences AD has on those affected and their caregivers, economical costs related to the disease are massive.
% 
% One of the difficulties for successful treatment of AD is the fact that its pathological hallmarks tend to be established in the brain decades prior to the time a person's cognitive and functional impairments are severe enough to get medical attention. Management of known risk factors for AD (e.g., high blood pressure and diabetes) is therefore emphasized. Recent studies point towards promising life-style interventions reducing AD-pathology and neurodegeneration and delaying symptom-onset. Much effort is therefore put into early identification and treatment of patients in the prodromal phase of the disease. Mild Cognitive Impairment (MCI) has become a diagnostic concept to describe this phase. Individuals falling within this diagnostic category show a cognitive decline greater than expected in normal cognitive aging, but still not with the severity of functional impairment characterizing those with dementia.


% \color{yellow}{\tiny For full guidelines regarding your manuscript please refer to \href{http://www.frontiersin.org/about/AuthorGuidelines}{Author Guidelines}.

% As a primary goal, the abstract should render the general significance and conceptual advance of the work clearly accessible to a broad readership. References should not be cited in the abstract. Leave the Abstract empty if your article does not require one, please see \href{http://www.frontiersin.org/about/AuthorGuidelines#SummaryTable}{Summary Table} for details according to article type.

% } 


\tiny
 \keyFont{ \section{Keywords:} keyword, keyword, keyword, keyword, keyword, keyword, keyword, keyword} %All article types: you may provide up to 8 keywords; at least 5 are mandatory.
\end{abstract}

\section{Each patient is unique}

%%%% EXAMPLE-PATIENT DATA %%%%
% Olivia, Ariel, Bianca:
% Apoe4_
% 1
% ANARTERR_neuro
% 14
% AVDEL30MIN_neuro
% 3
% AVDELTOT_neuro
% 12
% CATANIMSC_neuro
% 18
% GDTOTAL_gds
% 0
% RAVLT_immediate
% 29
% AGE
% 83.4873
% LRHHC_n_long
% 0.00276752
% TRAASCOR_neuro
% 126.543
% TRABSCOR_neuro
% 117.003
% Gender_num_
% 1
%
%%%%%%%%%
% Curtis:
% Apoe4_
% 0
% ANARTERR_neuro
% 6
% AVDEL30MIN_neuro
% 0
% AVDELTOT_neuro
% 2
% CATANIMSC_neuro
% 13
% GDTOTAL_gds
% 1
% RAVLT_immediate
% 16
% AGE
% 61.517
% LRHHC_n_long
% 0.00416919
% TRAASCOR_neuro
% 24.8009
% TRABSCOR_neuro
% 83.4148
% Gender_num_
% 0
%
%%%%%%%%
% Prior prob. for Bianca: 0.3
%%%%%%%%
% Utility matrices Olivia, Bianca, Curtis:
% 1  1.0  0.0
% 2  0.6  0.7
% 3  0.0  1.0
%
% Utility matrix Ariel:
% 1  1.0  0.0
% 2  0.7  0.5
% 3  0.0  1.0
% 4  0.5  0.8

Meet Olivia, Ariel, Bianca, Curtis.\footnote{Fictive characters; any reference to real persons is purely coincidental} These four persons don't know each other, but they have something in common: they all suffer from a mild form of cognitive impairment, and are afraid that their impairment will turn into Alzheimer's disease within a couple of years. In fact, this is why they recently underwent some clinical analyses and cognitive tests. Today they got the results of their analyses. From these results and other demographic factors their clinician is assessing their risk of developing Alzheimer, and will then decide on possible preventive treatments together with the patients.

Besides this shared condition and worry, these patients have other things in common -- but also some differences. Let's take Olivia as reference and see the similarities and difference between her and the other three:
\begin{itemize}
\item Olivia and Ariel turn out to have exactly identical clinical results, age, and geographical origin, and very similar family histories. Ariel, however, is soon going to move to another country where a new preventive treatment is available; this option is not open to Olivia.
\item Olivia and Bianca also have exactly identical clinical results and age, and the same preventive options. Bianca, however, comes from a different geographical region, having lower conversion rate, 30\%, from Mild Cognitive Impairment to Alzheimer. Moreover there is no history of Alzheimer in her family.
\item Olivia and Curtis have different clinical results and age -- in particular, Olivia has the risky APOE4 allele whereas Curtis hasn't, and Curtis is more than 20 years younger. But they otherwise come from the same geographical region, have very similar family histories, and the same preventive options.
\end{itemize}
Figure*** summarizes the similarity and differences between Olivia and the other three patients. Table*** reports the clinical results and demographic data common to Olivia, Ariel, Bianca; and those of Curtis.






\newpage
\hrule
\hrule

--- Luca, old pieces of text ---

Personalized diagnosis, prognosis, treatment, and prevention strategies must make allowance for several fundamental differences among patients:
\begin{itemize}
\item the \emph{kinds} of clinical data and evidence available for diagnosis or prognosis can be different;
\item the \emph{values} of the same kind of clinical data and evidence can be different;
\item the kinds of treatment or prevention options can be different;
\item the advantages and disadvantages, benefits and costs of the same kinds of treatment or prevention can be different;
\item finally, the evaluation of the clinician -- which often relies on too subtle clues (family history, regional history, case experience) to be considered as measurable data -- can be different.
\end{itemize}

Is there really a methodological framework that can take all these differences into account? Yes, there is, and Medicine has the distinction of having been one of the first fields to adopt it \citep{ledleyetal1959}: Statistical Decision Theory. Its application in Medicine is explained and exemplified in several, brilliant, old and new textbooks \citep{weinsteinetal1980,soxetal1988_r2013,huninketal2001_r2014}. This theory has mathematical and logical foundations and its principles constitute indeed the foundations for the definition and realization of Artificial Intelligence \citep{russelletal1995_r2022}
\mynotew{}



The basics of clinical decision making \mynotew{..basics: each piece of evidence contributes with a likelihood or odds; they combine together and together with the clinician's pre-data evaluation. Then they are combined with the different benefits/costs of treatments or prevention strategies to find the optimal one. Decision trees can be necessary (but don't change this framework). Costs \& benefits are evaluated by clinician \& patient together.}


\begin{multline}\label{eq:bayes_theorem_intro}
    \overbrace{\vphantom{\bigg\{}\p(\text{health condition}\|
      \text{results of all tests},\ 
      \text{prior info})}^{\zerob{\scriptsize\emph{post-test probability}}}
    \mathrel{\ \propto\ }{}
    \\
    \shoveright{\overbrace{\vphantom{\bigg\{}\p(\text{health condition} \|
      \text{prior info})}^{\zerob{\scriptsize\emph{pre-test probability by clinician}}}
  \mathbin{\ \times\ }{}}    
    \\[\jot]
\text{\scriptsize\emph{likelihoods of tests}}\left\{\;  \begin{aligned}
 \p(\text{result of 1st test} \| \text{health condition} ,\ 
  \text{prior info})
  &\mathbin{\ \times\ }{}\\
  \p(\text{result of 2nd test} \| \text{health condition} ,\ 
  \text{prior info})
  &\mathbin{\ \times\ }{}
\\
{}\dotsb &
  \end{aligned}\right.
\end{multline}


{\color{yellow}\tiny For Original Research Articles \citep{conference}, Clinical Trial Articles \citep{article}, and Technology Reports \citep{patent}, the introduction should be succinct, with no subheadings \citep{book}. For Case Reports the Introduction should include symptoms at presentation \citep{chapter}, physical exams and lab results \citep{dataset}.

}



\section{Article types}

For requirements for a specific article type please refer to the Article Types on any Frontiers journal page. Please also refer to  \href{http://home.frontiersin.org/about/author-guidelines#Sections}{Author Guidelines} for further information on how to organize your manuscript in the required sections or their equivalents for your field

% For Original Research articles, please note that the Material and Methods section can be placed in any of the following ways: before Results, before Discussion or after Discussion.

\section{Manuscript Formatting}

\subsection{Heading Levels}

%There are 5 heading levels

\subsection{Level 2}
\subsubsection{Level 3}
\paragraph{Level 4}
\subparagraph{Level 5}

\subsection{Equations}
Equations should be inserted in editable format from the equation editor.

\begin{equation}
\sum x+ y =Z\label{eq:01}
\end{equation}

\subsection{Figures}
Frontiers requires figures to be submitted individually, in the same order as they are referred to in the manuscript. Figures will then be automatically embedded at the bottom of the submitted manuscript. Kindly ensure that each table and figure is mentioned in the text and in numerical order. Figures must be of sufficient resolution for publication \href{https://www.frontiersin.org/about/author-guidelines#ImageSizeRequirements}{see here for examples and minimum requirements}. Figures which are not according to the guidelines will cause substantial delay during the production process. Please see \href{https://www.frontiersin.org/about/author-guidelines#FigureRequirementsStyleGuidelines}{here} for full figure guidelines. Cite figures with subfigures as figure \ref{fig:Subfigure 1} and \ref{fig:Subfigure 2}.


\subsubsection{Permission to Reuse and Copyright}
Figures, tables, and images will be published under a Creative Commons CC-BY licence and permission must be obtained for use of copyrighted material from other sources (including re-published/adapted/modified/partial figures and images from the internet). It is the responsibility of the authors to acquire the licenses, to follow any citation instructions requested by third-party rights holders, and cover any supplementary charges.
%%Figures, tables, and images will be published under a Creative Commons CC-BY licence and permission must be obtained for use of copyrighted material from other sources (including re-published/adapted/modified/partial figures and images from the internet). It is the responsibility of the authors to acquire the licenses, to follow any citation instructions requested by third-party rights holders, and cover any supplementary charges.

\subsection{Tables}
Tables should be inserted at the end of the manuscript. Please build your table directly in LaTeX.Tables provided as jpeg/tiff files will not be accepted. Please note that very large tables (covering several pages) cannot be included in the final PDF for reasons of space. These tables will be published as \href{http://home.frontiersin.org/about/author-guidelines#SupplementaryMaterial}{Supplementary Material} on the online article page at the time of acceptance. The author will be notified during the typesetting of the final article if this is the case. 

\section{Nomenclature}

\subsection{Resource Identification Initiative}
To take part in the Resource Identification Initiative, please use the corresponding catalog number and RRID in your current manuscript. For more information about the project and for steps on how to search for an RRID, please click \href{http://www.frontiersin.org/files/pdf/letter_to_author.pdf}{here}.

\subsection{Life Science Identifiers}
Life Science Identifiers (LSIDs) for ZOOBANK registered names or nomenclatural acts should be listed in the manuscript before the keywords. For more information on LSIDs please see \href{https://www.frontiersin.org/about/author-guidelines#Nomenclature}{Inclusion of Zoological Nomenclature} section of the guidelines.


\section{Additional Requirements}

For additional requirements for specific article types and further information please refer to \href{http://www.frontiersin.org/about/AuthorGuidelines#AdditionalRequirements}{Author Guidelines}.

\section*{Conflict of Interest Statement}
%All financial, commercial or other relationships that might be perceived by the academic community as representing a potential conflict of interest must be disclosed. If no such relationship exists, authors will be asked to confirm the following statement: 

The authors declare that the research was conducted in the absence of any commercial or financial relationships that could be construed as a potential conflict of interest.

\section*{Author Contributions}

The Author Contributions section is mandatory for all articles, including articles by sole authors. If an appropriate statement is not provided on submission, a standard one will be inserted during the production process. The Author Contributions statement must describe the contributions of individual authors referred to by their initials and, in doing so, all authors agree to be accountable for the content of the work. Please see  \href{https://www.frontiersin.org/about/policies-and-publication-ethics#AuthorshipAuthorResponsibilities}{here} for full authorship criteria.

\section*{Funding}
Details of all funding sources should be provided, including grant numbers if applicable. Please ensure to add all necessary funding information, as after publication this is no longer possible.

\section*{Acknowledgments}
This is a short text to acknowledge the contributions of specific colleagues, institutions, or agencies that aided the efforts of the authors.

\section*{Supplemental Data}
 \href{http://home.frontiersin.org/about/author-guidelines#SupplementaryMaterial}{Supplementary Material} should be uploaded separately on submission, if there are Supplementary Figures, please include the caption in the same file as the figure. LaTeX Supplementary Material templates can be found in the Frontiers LaTeX folder.

\section*{Data Availability Statement}
The datasets [GENERATED/ANALYZED] for this study can be found in the [NAME OF REPOSITORY] [LINK].
% Please see the availability of data guidelines for more information, at https://www.frontiersin.org/about/author-guidelines#AvailabilityofData

\bibliographystyle{Frontiers-Harvard} %  Many Frontiers journals use the Harvard referencing system (Author-date), to find the style and resources for the journal you are submitting to: https://zendesk.frontiersin.org/hc/en-us/articles/360017860337-Frontiers-Reference-Styles-by-Journal. For Humanities and Social Sciences articles please include page numbers in the in-text citations 
%\bibliographystyle{Frontiers-Vancouver} % Many Frontiers journals use the numbered referencing system, to find the style and resources for the journal you are submitting to: https://zendesk.frontiersin.org/hc/en-us/articles/360017860337-Frontiers-Reference-Styles-by-Journal
\bibliography{test,portamanabib}

%%% Make sure to upload the bib file along with the tex file and PDF
%%% Please see the test.bib file for some examples of references

\section*{Figure captions}

%%% Please be aware that for original research articles we only permit a combined number of 15 figures and tables, one figure with multiple subfigures will count as only one figure.
%%% Use this if adding the figures directly in the mansucript, if so, please remember to also upload the files when submitting your article
%%% There is no need for adding the file termination, as long as you indicate where the file is saved. In the examples below the files (logo1.eps and logos.eps) are in the Frontiers LaTeX folder
%%% If using *.tif files convert them to .jpg or .png
%%%  NB logo1.eps is required in the path in order to correctly compile front page header %%%

\begin{figure}[h!]
\begin{center}
\includegraphics[width=10cm]{logo1}% This is a *.eps file
\end{center}
\caption{ Enter the caption for your figure here.  Repeat as  necessary for each of your figures}\label{fig:1}
\end{figure}

\setcounter{figure}{2}
\setcounter{subfigure}{0}
\begin{subfigure}
\setcounter{figure}{2}
\setcounter{subfigure}{0}
    \centering
    \begin{minipage}[b]{0.5\textwidth}
        \includegraphics[width=\linewidth]{logo1.eps}
        \caption{This is Subfigure 1.}
        \label{fig:Subfigure 1}
    \end{minipage}  
   
\setcounter{figure}{2}
\setcounter{subfigure}{1}
    \begin{minipage}[b]{0.5\textwidth}
        \includegraphics[width=\linewidth]{logo2.eps}
        \caption{This is Subfigure 2.}
        \label{fig:Subfigure 2}
    \end{minipage}

\setcounter{figure}{2}
\setcounter{subfigure}{-1}
    \caption{Enter the caption for your subfigure here. \textbf{(A)} This is the caption for Subfigure 1. \textbf{(B)} This is the caption for Subfigure 2.}
    \label{fig: subfigures}
\end{subfigure}

%%% If you don't add the figures in the LaTeX files, please upload them when submitting the article.
%%% Frontiers will add the figures at the end of the provisional pdf automatically
%%% The use of LaTeX coding to draw Diagrams/Figures/Structures should be avoided. They should be external callouts including graphics.

\end{document}
